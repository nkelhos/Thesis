\cleartooddpage[\thispagestyle{empty}]
\chapter{Analysis}

\section{Veritas Data}
runs/sources/dates/times

\section{Likelihood Ratio Test}
The likelihood ratio test useful for comparing two hypotheses.
They are referred to as the null and alternative hypotheses.
Each hypothesis consists of a predicted number of events at each point in the energy/space/time parameter space.
Once each hypothesis is constructed, the likelihood for each can be computed.
The ratio of the two likelihoods then follows a gaussian (with certain assumptions), meaning the sigificance of the alternative hypothesis over the null can be calculated.

\section{Backgrounds}

Backgrounds were from data.

\section{Test Sources}

  \subsection{Crab}
    Only studied runs with elevations between 70 and 75 degrees.
    Used Segue 1 data at the same elevations as a background.

    \subsubsection{Event Display}
    \subsubsection{CTOOLs}

  \subsection{HESS J0632 +057}
    Only studied data with pointing elevations between 59 and 65 degrees.
    Used Draco data at the same elevations as the background.

    \subsubsection{Event Display}
    \subsubsection{CTOOLs}

  \subsection{1ES 0414+009}
    Only studied data with pointing elevations between 53 and 60 degrees.
    Used Ursa Major II data at the same elevations as the background.

    \subsubsection{Event Display}
    \subsubsection{CTOOLs}

  \subsection{1ES1959+650}
    Only studied data with pointing elevations between 50 and 55 degrees.
    Used Ursa Minor data at the same elevations as the background.

    \subsubsection{Event Display}
    \subsubsection{CTOOLs}

\section{Sgr A* Data Analysis}
Broke data into two elevation regimes, 25-28.5 degrees, and 28.5 to 30.25 degrees.
Separate off runs were taken to use as a background.

  \subsubsection{Event Display}

\section{Astrophysical Models}

\subsection{Point Sources}

\subsection{Supernovas}

\subsection{Diffuse Emission}

\subsection{Dark Matter}

\section{Upper Limit}

comparison to 

