\cleartooddpage[\thispagestyle{empty}]
\chapter{Analysis}

\section{Studies}
Backgrounds from data were analyzed.

\subsection{Low Elevations}
For VERITAS, the galactic center is at a low elevation, usually around 29 degrees.
As part of the observing strategy for the galactic center, off data is also taken near the galactic center.
This means that, for most nights when the galactic center is observed (called On runs, at RA/Dec 260.91/-29.01), one run is also taken with the observatory pointing a few degrees away from the galactic center, an Off run at RA/Dec 266.41/-29.01 .
This is done because the galactic center is a highly active area, meaning there are few regions within the On data that can be used as a background region in the Li and Ma significance equation ??.

Data in two different elevation brackets.

Shows Cresecent pattern at lowest energies, doughnut at slightly higher energies, then at higher energies is concentrated at the lower part of the camera.

\subsection{Diffuse Gamma Rays}

Sets gamma rays were simulated, with corsika, at different energies.
These gamma rays were diffused in a circular disk of radius ?? degrees around ?? elevation/azimuth.
The gamma rays were then processed through the VERITAS simulation chain.
This simulation chain takes the cherenkov photons from each gamma ray shower, and calculates which ones hit telescope mirrors.
Then, it calculates which cherenkov photons hit which pixels, and at what time.
Once cherenkov photons are measured hitting a pixel, a hardware simulation program takes over to simulate the voltage pulse created by the photons.
This voltage pulse is then propagated through a software simulation of the VERITAS triggering system.
Once the list of triggered gamma-ray showers is save to a file, reconstruction of the event can be done by the same software that reconstructs observed events.

\subsection{Elevation Effects}

Diffuse sims with direct-camera pointing show just the elevation effects, no camera effects.

Elevation applies a different gradient depending on energy.

\subsection{Gamma Verses Protons}

Doughnut shapes appear in the gamma-like protons.


\section{Veritas Data}
runs/sources/dates/times

\section{Likelihood Ratio Test}
The likelihood ratio test useful for comparing two hypotheses.
They are referred to as the null and alternative hypotheses.
Each hypothesis consists of a predicted number of events at each point in the energy/space/time parameter space.
Once each hypothesis is constructed, the likelihood for each can be computed.
The ratio of the two likelihoods then follows a gaussian (with certain assumptions), meaning the sigificance of the alternative hypothesis over the null can be calculated.

\section{Backgrounds}

Backgrounds were from data.

\section{Test Sources}

  \subsection{Crab}
    Only studied runs with elevations between 70 and 75 degrees.
    Used Segue 1 data at the same elevations as a background.

    \subsubsection{Event Display}
    \subsubsection{CTOOLs}

  \subsection{HESS J0632 +057}
    Only studied data with pointing elevations between 59 and 65 degrees.
    Used Draco data at the same elevations as the background.

    \subsubsection{Event Display}
    \subsubsection{CTOOLs}

  \subsection{1ES 0414+009}
    Only studied data with pointing elevations between 53 and 60 degrees.
    Used Ursa Major II data at the same elevations as the background.

    \subsubsection{Event Display}
    \subsubsection{CTOOLs}

  \subsection{1ES1959+650}
    Only studied data with pointing elevations between 50 and 55 degrees.
    Used Ursa Minor data at the same elevations as the background.

    \subsubsection{Event Display}
    \subsubsection{CTOOLs}

\section{Sgr A* Data Analysis}
Broke data into two elevation regimes, 25-28.5 degrees, and 28.5 to 30.25 degrees.
Separate off runs were taken to use as a background.

  \subsubsection{Event Display}

\section{Astrophysical Models}

\subsection{Point Sources}

\subsection{Supernovas}

\subsection{Diffuse Emission}

\subsection{Dark Matter}

\section{Upper Limit}

comparison to 

