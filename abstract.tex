\cleartooddpage[\thispagestyle{empty}]
\phantomsection
\section*{Abstract}
\addcontentsline{toc}{chapter}{Abstract}

VERITAS is a Gamma-Ray Observatory that has been observing astrophysical gamma-ray targets like the Galactic Center for several years.
As VERITAS is in the Northern Hemisphere, this means the telescope detects Gamma Rays at a low elevation.
This leads to the telescope being sensitive to higher energy gamma rays than if it were pointed directly overhead.
As the Galactic Center is at low elevations and is a prime target for Dark Matter searches, a probe of Dark Matter at unsearched energies is possble.
This dissertation presents the analysis methods used to search for Dark Matter near the Galactic Center from gamma rays with energies between 4 and 70 TeV.


%% add a blank page for margin styling
%\newpage
%\null
%\thispagestyle{empty}
%\newpage

\cleartoevenpage[\thispagestyle{plain}]
\null
