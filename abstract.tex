\cleartooddpage[\thispagestyle{empty}]
\phantomsection
\section*{Abstract}
\addcontentsline{toc}{chapter}{Abstract}

Dark matter accounts for 24\% of the universe's energy, but the form it is stored in is currently unknown.
Understanding what form this matter takes is one of the major unsolved mysteries of modern physics.
Much evidence exists for dark matter, in the measurements of dwarf galaxies, galaxies, galaxy clusters, and cosmological measurements.
Dark matter itself is posited to be a new undiscovered particle that only interacts via gravity and the weak force, called a WIMP.
This WIMP takes the form of a supersymmetric particle called a Neutralino.
The objective of this thesis is to search for these dark matter particles, and attempt to identify their mass and cross section.

Dark matter particles appear to concentrate in most galaxy-scale gravitational wells.
One region of space that is both nearby and has a high density of dark matter is the center of our own galaxy.
Additionally, if two dark matter particles annihilate into Standard Model particles, they should indirectly produce a spectrum of gamma rays.
Therefore, a search for gamma rays near the Galactic Center may uncover the presence of dark matter.

Gamma rays can be detected at Earth by observing the particle shower produced when a gamma ray strikes Earth's atmosphere.
This produces a shower of electrons, positrons, and photons, called an air shower.
The electrons and positrons travel faster than the speed of light in the atmosphere, which create Cherenkov photons in the atmosphere.
A specialized telescope consisting of mirrors and photomultiplier tubes can then detect these Cherenkov photons.
By observing the Cherenkov photons, an image can be produced of the original air shower that was created by a gamma ray.
This image can then be used to reconstruct the original gamma-ray's energy and point-of-origin.
By utilizing two or more linked telescopes, multiple images of each air shower can be taken, to further improve the accuracy when reconstructing the energy and point-of-origin of the gamma ray.

VERITAS is a gamma-ray observatory located in southern Arizona, that consists of four of these telescopes linked together.
For several years, VERITAS has been observing gamma rays originating from within several degrees of the Galactic Center, and has accumulated \SI{108}{hours} of data.
This data contains a list of gamma rays and VERITAS detector information, and is used in a likelihood analysis.
Due to the latitude of the VERITAS observatory and the position of the Galactic Center in the sky, VERITAS must point its telescopes at a low elevation (\nicetilde\ang{28}) in order to observe the Galactic Center.
This alters the sensitivity of VERITAS when detecting gamma rays.
Specifically, this shifts the sensitive energy range upwards, from \SIrange{0.05}{50}{\TeV} to \SIrange{1.5}{70}{\TeV{}}.
This low elevation also affects the distribution of events in the camera, as the atmospheric air-mass column density is 12\% higher at the bottom of the camera than at the top.
To limit the effects of this in the dark matter analysis, gamma rays are limited to \SIrange{4}{70}{\TeV}.

The unbinned likelihood analysis used in this thesis consists of modeling the cloud of dark matter around the Galactic Center, as well as the spectrum of gamma rays produced when two WIMPs annihilate.
An additional point source is added, to model the non-dark-matter gamma-ray emission that is detected from the Galactic Center.
Background models are also constructed using several hours of off-Galactic-Center data, to account for the many proton-induced air showers that form an irreducible background in the gamma-ray data.
This dissertation presents the analysis methods used to search for WIMP dark matter at 9 different masses near the Galactic Center with \TeV{} gamma rays.

{\color{red}(how does this abstract look??)}

%% add a blank page for margin styling
%\newpage
%\null
%\thispagestyle{empty}
%\newpage

\cleartoevenpage[\thispagestyle{plain}]
\null
