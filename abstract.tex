\cleartooddpage[\thispagestyle{empty}]
\phantomsection
\section*{Abstract}
\addcontentsline{toc}{chapter}{Abstract}

Dark matter accounts for 24\% of the universe's energy, but the form in which it is stored in is currently unknown.
Understanding what form this matter takes is one of the major unsolved mysteries of modern physics.
Much evidence exists for dark matter in the measurements of galaxies, dwarf galaxies, galaxy clusters, and cosmological measurements.
One theory posits Dark matter is a new undiscovered particle that only interacts via gravity and the weak force, called a \textit{weakly interacting massive particle} (WIMP).
One WIMP candidate is a supersymmetric particle called a Neutralino {\color{red}(better??)}.
The objective of this thesis is to search for these dark matter particles, and attempt to measure their mass and cross section.

Dark matter particles appear to concentrate in most galaxy-scale gravitational wells.
One region of space that is both nearby and assumed to have a high density of dark matter is the center of our own galaxy.
If two dark matter particles annihilate into Standard Model particles, then many annihilations should indirectly produce a spectrum of gamma rays {\color{red}(better??)}.
Therefore, a search for gamma rays near the Galactic Center may uncover the presence of dark matter.

VERITAS is a gamma-ray observatory located in southern Arizona.
For several years, VERITAS has taken \SI{108}{hours} of low-elevation Galactic Center data.
This data contains a list of gamma rays and VERITAS detector information, which is used in a likelihood analysis.
The low-elevation nature of the data shifts VERITAS's sensitive energy range from \SIrange{0.05}{50}{\TeV{}} to \SIrange{1.5}{70}{\TeV{}}.
Due to low-elevation atmospheric effects, the energy range is further limited to \SIrange{4}{70}{\TeV{}}.

The unbinned likelihood analysis used in this thesis consists of modeling the halo of dark matter at the Galactic Center, as well as the spectrum of gamma rays produced when two WIMPs annihilate.
An additional point source is added to model the non-dark-matter gamma-ray emission that is detected from the Galactic Center.
Background models are also constructed from separate off-Galactic-Center observations.

The results of this analysis are that no dark matter signal was found at any of the 9 WIMP masses tested.
Additionally, upper limits on the WIMP's velocity-averaged cross section have been calculated, which result in new limits above \SI{70}{\frac{\TeV{}}{c^2}} at the 95\% confidence level.
This dissertation presents the analysis methods used to search for these WIMP signals.

{\color{red}(abstract better??)}


%Background models are also constructed using several hours of off-Galactic-Center data, to account for the many proton-induced air showers that form an irreducible background in the gamma-ray data.

%The low-elevation nature of the data affects VERITAS's sensitive energy range and adds due to the atmospheric density, from \SIrange{0.05}{50}{\TeV} to \SIrange{1.5}{70}{\TeV{}}.
%The energy range was further limited to \SIrange{4}{70}{\TeV{}} to the effects of 

%Due to the latitude of the VERITAS observatory and the position of the Galactic Center in the sky, VERITAS must point its telescopes at a low elevation (\nicetilde\ang{28}) in order to observe the Galactic Center.
%This alters the sensitivity of VERITAS when detecting gamma rays.
%This low elevation also affects the distribution of events in the camera, as the atmospheric air-mass column density is 12\% higher at the bottom of the camera than at the top.
%To limit the effects of this in the dark matter analysis, gamma rays are limited to \SIrange{4}{70}{\TeV}.

%Gamma rays can be detected at Earth by observing the particle shower produced when a gamma ray strikes Earth's atmosphere.
%This produces a shower of electrons, positrons, and photons, which is called an air shower.
%The electrons and positrons travel faster than the speed of light in the atmosphere, which create Cherenkov photons in the atmosphere.
%A specialized telescope consisting of mirrors and photomultiplier tubes can then detect these Cherenkov photons.
%By observing the Cherenkov photons, an image can be produced of the original air shower that was created by a gamma ray.
%This image can then be used to reconstruct the original gamma-ray's energy and point-of-origin.
%By utilizing two or more linked telescopes, multiple images of each air shower can be taken, to further improve the accuracy when reconstructing the energy and point-of-origin of the gamma ray.

%% add a blank page for margin styling
%\newpage
%\null
%\thispagestyle{empty}
%\newpage

\cleartoevenpage[\thispagestyle{plain}]
\null
