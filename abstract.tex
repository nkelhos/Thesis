\cleartooddpage[\thispagestyle{empty}]
\phantomsection
\section*{Abstract}
\addcontentsline{toc}{chapter}{Abstract}

Dark matter accounts for 24\% of the universe's energy, but the form in which it is stored in is currently unknown.
Understanding what form this matter takes is one of the major unsolved mysteries of modern physics.
Much evidence exists for dark matter in the measurements of galaxies, dwarf galaxies, galaxy clusters, and cosmological measurements.
Dark matter {\color{red}itself is posited to be (only one of the options, you can write one of the more accepted if you want?? -orel)}a new undiscovered particle that only interacts via gravity and the weak force, called a \textit{weakly interacting massive particle} (WIMP).
This WIMP takes the form of a supersymmetric particle called a Neutralino {\color{red}(this is only one of the options?? -orel)}.
The objective of this thesis is to search for these dark matter particles, and attempt to measure their mass and cross section.

Dark matter particles appear to concentrate in most galaxy-scale gravitational wells.
One region of space that is both nearby and assumed to have a high density of dark matter is the center of our own galaxy.
If two dark matter particles annihilate into Standard Model particles, they should indirectly produce a {\color{red}spectrum (it is weird to write that two particles would produce a spectrum of gamma rays?? -orel)} of gamma rays.
Therefore, a search for gamma rays near the Galactic Center may uncover the presence of dark matter.

Gamma rays can be detected at Earth by observing the particle shower produced when a gamma ray strikes Earth's atmosphere.
This produces a shower of electrons, positrons, and photons, which is called an air shower.
The electrons and positrons travel faster than the speed of light in the atmosphere, which create Cherenkov photons in the atmosphere.
A specialized telescope consisting of mirrors and photomultiplier tubes can then detect these Cherenkov photons.
By observing the Cherenkov photons, an image can be produced of the original air shower that was created by a gamma ray.
This image can then be used to reconstruct the original gamma-ray's energy and point-of-origin.
By utilizing two or more linked telescopes, multiple images of each air shower can be taken, to further improve the accuracy when reconstructing the energy and point-of-origin of the gamma ray.

VERITAS is a gamma-ray observatory located in southern Arizona which consists of four of these telescopes linked together.
For several years, VERITAS has been observing gamma rays originating from within several degrees of the Galactic Center, and has accumulated \SI{108}{hours} of data.
This data contains a list of gamma rays and VERITAS detector information, and is used in a likelihood analysis.
Due to the latitude of the VERITAS observatory and the position of the Galactic Center in the sky, VERITAS must point its telescopes at a low elevation (\nicetilde\ang{28}) in order to observe the Galactic Center.
This alters the sensitivity of VERITAS when detecting gamma rays.
Specifically, this shifts the sensitive energy range upwards, from \SIrange{0.05}{50}{\TeV} to \SIrange{1.5}{70}{\TeV{}}.
This low elevation also affects the distribution of events in the camera, as the atmospheric air-mass column density is 12\% higher at the bottom of the camera than at the top.
To limit the effects of this in the dark matter analysis, gamma rays are limited to \SIrange{4}{70}{\TeV}.

The unbinned likelihood analysis used in this thesis consists of modeling the {\color{red}cloud of dark matter around the Galactic Center (not sure if this is correct when actually the highest concentration of DM is at the location of the GC?? -gernot)}, as well as the spectrum of gamma rays produced when two WIMPs annihilate.
An additional point source is added to model the non-dark-matter gamma-ray emission that is detected from the Galactic Center.
Background models are also constructed using several hours of off-Galactic-Center data, to account for the many proton-induced air showers that form an irreducible background in the gamma-ray data.
This dissertation presents the analysis methods used to search for WIMP dark matter at 9 different masses near the Galactic Center with \TeV{} gamma rays.

{\color{red}(shorten abstract to max 1 page?? -gernot)}
{\color{red}(add some results (e.g. that you excluded a certain parameter space with xx confidence) ?? -gernot)}

{\color{red}(starting from the third paragraph, the abstract starts to become too detailed and not 'abstract-like'.
You do not need to explain how we detect gamma rays.  
You shouldnt give too many details on the analysis and you must present your results and discuss how they compare to other measurements.
As an example, it is enough to say "108 hours of low elevation data were used to search for a dark matter signal", "the energy range for the gamma rays owas limited to 4-70 TeV to avoid atmospheric effects at low elevation", and "The background was estimated from separate off-galactic-center observations".
That is the level of detail you need, I don't mean for you to use these sentences literally.
Please take a look at an abstract from a different thesis to get more examples/ideas (marias or moritzs) ?? -orel)}


%% add a blank page for margin styling
%\newpage
%\null
%\thispagestyle{empty}
%\newpage

\cleartoevenpage[\thispagestyle{plain}]
\null
