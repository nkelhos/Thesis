\cleartooddpage[\thispagestyle{empty}]
\phantomsection
\section*{Abstract}
\addcontentsline{toc}{chapter}{Abstract}

The High-Altitude Water Cherenkov (HAWC) Experiment is a gamma-ray observatory that utilizes water silos as Cherenkov detectors to measure the electromagnetic air showers created by gamma rays.
The experiment consists of an array of closely packed water Cherenkov detectors (WCDs), each with four photomultiplier tubes (PMTs).
The direction of the gamma ray will be reconstructed using the times when the electromagnetic shower front triggers PMTs in each WCD.
To achieve an angular resolution as low as 0.1 degrees, a laser calibration system will be used to measure relative PMT response times.
The system will direct 300$\,$ps laser pulses into two fiber-optic networks.
Each network will use optical fan-outs and switches to direct light to specific WCDs.
The first network is used to measure the light transit time out to each pair of detectors, and the second network sends light to each detector, calibrating the response times of the four PMTs within each detector.
As the relative PMT response times are dependent on the number of photons in the light pulse, neutral density filters will be used to control the light intensity across five orders of magnitude.
This system will run both continuously in a low-rate mode, and in a high-rate mode with many intensity levels.
In this thesis, the design of the calibration system and systematic studies verifying its performance are presented.


%% add a blank page for margin styling
%\newpage
%\null
%\thispagestyle{empty}
%\newpage

\cleartoevenpage[\thispagestyle{plain}]
\null
