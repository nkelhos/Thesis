\cleartooddpage[\thispagestyle{empty}]
\phantomsection
\section*{Abstract}
\addcontentsline{toc}{chapter}{Abstract}

Dark matter accounts for 24\% of the universe's energy, but the form in which it is stored is currently unknown.
Understanding what form this matter takes is one of the major unsolved mysteries of modern physics.
Much evidence exists for dark matter in the measurements of galaxies, dwarf galaxies, galaxy clusters, and cosmological measurements.
One theory posits dark matter is a new undiscovered particle that only interacts via gravity and the weak force, called a \textit{weakly interacting massive particle} (WIMP).
One WIMP candidate is a supersymmetric particle called a neutralino.
The objective of this thesis is to search for these dark matter particles, and attempt to measure their mass and cross section.

Dark matter particles appear to concentrate in most galaxy-scale gravitational wells.
One region of space that is both nearby and assumed to have a high density of dark matter is the center of our own galaxy.
The neutralino is expected to annihilate into Standard Model particles, which may decay into photons.
Therefore, a search for gamma rays near the Galactic Center may uncover the presence of dark matter.

108 hours of VERITAS gamma-ray observations of the Galactic Center are used in an unbinned likelihood analysis to search for dark matter.
The Galactic Center's low elevation results in VERITAS observing gamma rays in the \SIrange{4}{70}{\TeV{}} energy range.
The analysis used in this thesis consists of modeling the halo of dark matter at the Galactic Center, as well as the spectrum of gamma rays produced when two WIMPs annihilate.
A point source is added to model the non-dark-matter gamma-ray emission detected from the Galactic Center.
Background models are constructed from data of separate off-Galactic-Center observations.

No dark matter signal is found in the \SIrange{4}{100}{\TeV} mass range.
Upper limits on the WIMP's velocity-averaged cross section have been calculated, which above \SI{70}{\TeV{}} result in new limits of $\vacs{} < \left ( 6.6-7.6 \right ) \times 10^{-25} \frac{\mathrm{cm}^3}{\mathrm{s}}$ at the 95\% confidence level.


%% add a blank page for margin styling
%\newpage
%\null
%\thispagestyle{empty}
%\newpage

\cleartoevenpage[\thispagestyle{plain}]
\null
