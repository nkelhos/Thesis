\cleartooddpage[\thispagestyle{empty}]
\phantomsection
\section*{Abstract}
\addcontentsline{toc}{chapter}{Abstract}

VERITAS is a Gamma Ray Observatory that has been observing the gamma rays from the Galactic Center for several years.
As VERITAS is in the Northern Hemisphere, this means the telescope detects Gamma Rays at a low elevation.
This leads to the telescope being sensitive to higher energy gamma rays than if it were pointed directly overhead.
As the galactic center is a prime target for Dark Matter searches, combined with the low elevation telescope observations, allows for a probe of Dark Matter at unique energies, relatively unconstrained by other telescopes.
As the Galactic Center contains many different sources of gamma rays, modelling each will be important in order to search for Dark Matter.
In this dissertation, the analysis methods used to search for Dark Matter near the Galactic Center at energies greater than 1 TeV are presented.


%% add a blank page for margin styling
%\newpage
%\null
%\thispagestyle{empty}
%\newpage

\cleartoevenpage[\thispagestyle{plain}]
\null
