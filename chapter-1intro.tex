\cleartooddpage[\thispagestyle{empty}]
\newcommand{\nicetilde}{{\raise.17ex\hbox{$\scriptstyle\mathtt{\sim}$}}}
\chapter{Introduction}

  How much mass is contained within a given volume of space?
  Within a galaxy?
  Within a cluster of galaxies?
  Within the observable Universe?

  These questions have been asked repeatedly since the 1930s~\cite{zwicky1937}.
  However, a variety of assessment techniques consistently result in very different mass measurements.
  Observing cosmological features like the cosmic microwave background, the distribution of galaxies, and gravitational lensing around galaxy clusters results in a larger mass, while observing the quantity of light produced results in a smaller mass.
  These observations occur over several different length scales, and are discussed in Chapter~\ref{ch_dm}.

  Dwarf galaxies, discussed in Section~\ref{dm_dwarfscale}, are gravitationally bound groups of stars that orbit their center of mass at a wider distribution of velocities than is expected from their luminous mass.
  Galaxy scale evidence is discussed in Section~\ref{dm_gal}, explaining how galaxies rotate faster than their baryonic mass alone would predict.
  Evidence on galaxy cluster scales is discussed in Section~\ref{dm_galclusters}.
  Galaxy Clusters have two clouds of mass, one of hydrogen gas observable via X-ray observations, another observable via the gravitational lensing of visible light.
  Collisions between two galaxy clusters cause these two pairs of clouds to pass through each other, dragging and slowing each other down at different rates.
  The differences in drag hint that the gravitationally-lensed mass has a smaller cross section than the hydrogen gas cross section.
  In Section~\ref{dm_universe}, Universe-scale evidence is discussed.
  In it, the observed spectrum of the cosmic microwave background matches predictions, but only when there is much more mass present in the Universe than is currently interacting with light.

  All of this heavily implies that there is missing mass, missing \textit{stuff}, unaccounted for by the existing model of physics and our Universe.
  As this mass prefers to only interact gravitationally and seemingly ignores the electromagnetic spectrum, it has earned the (in)conspicuous title of \textit{dark matter}.

\section{Motivation}
  This thesis will demonstrate that, by detecting gamma-rays from the Galactic Center with the VERITAS observatory, a search for dark matter can be conducted.
  Dark matter is proposed to be a particle that annihilates with itself into standard model particles.
  If dark matter particles form a spherical cloud (referred to as a halo) around the center of our galaxy, and if the annihliation of these particles produces (either directly or through secondary interactions) a detectable quantity of gamma rays, the density profile of this dark matter halo can be measured.
  If instead the gamma-ray flux is below detection limits, then observations and knowledge of VERITAS's sensitivity can help place an upper limit on the cross section of the dark matter particle.

\section{Dark Matter}

  % WMAP: 4.6% energy from baryonic matter, 24% energy from cold dark matter
  From measuring the cosmic microwave background, the WMAP~\cite{wmap9year_obs} satellite found that dark matter makes up 24\% of the energy of the Universe~\cite{pdg_2012}; therefore understanding the nature of dark matter is fundamental to understanding our Universe.
  A leading explanation is that dark matter is a new particle, not included in the standard model of particle physics.
  As many particle models predict dark matter to either a) couple weakly to other particles, b) interact via the Weak force, or c) both, it is referred to as a Weakly Interacting Massive Particle, or WIMP.
  The Standard Model, and the WIMP's place in it, are discussed in Section~\ref{sec_particledm}.

  From both cosmology and particle physics, the WIMP is predicted to have a mass in the range of GeV to TeV, and a velocity-averaged self-annihilation cross section of around \nicetilde{}\SI{3e-26}{cm${}^3$s${}^{-1}$}.
  In Chapter~\ref{ch_gamma}, the production of gamma rays from dark matter is discussed.
  WIMPs may directly annihilate or decay into gamma rays, or they may first produce quarks or leptons, which would then produce gamma rays through secondary interactions.
  These gamma rays would then have energies similar to the original WIMP mass, around the TeV scale.
  This potential for WIMP dark matter to produce TeV gamma rays makes it an attractive science target for gamma-ray observatories like VERITAS.
  A description of the VERITAS observatory and gamma-ray-detecting hardware is discussed in Chapter~\ref{chapter:veritas}.
  The method of reconstructing a gamma ray's position and energy are discussed in Chapter~\ref{ch:grrecon}.

  Then, the question becomes, where should we point our gamma-ray observatories?
  From gravitational and optical measurements, it is well documented that halos of dark matter augment the gravitational wells of most dwarf galaxies, regular galaxies, and galaxy clusters.
  Dwarf galaxies tend to have fewer background gamma-ray sources, but also have lower quantities of dark matter, making them weaker sources of gamma rays from dark matter annihilations.
  Galaxy Clusters, while more massive (and thus more emissive), have a non-negligible redshift, which introduces more model parameters.
  The Galactic Center, on the other hand, possesses higher densities of dark matter mass than any dwarf galaxy, while also being closer than any dwarf galaxy, making it an excellent target for a dark matter search.


\section{Galactic Center and Gamma Rays}

  At the center of our galaxy there is a supermassive black hole, with a mass of \SI{4e6}{ M${{}_\odot}$ }~\cite{sgra_massdist}.
  As dark matter appears to accompany most baryonic gravitational wells, it is expected there is a halo of dark matter particles at the Galactic Center.
  Studying this halo is difficult, however, as our galaxy's large gravitational well has accreted a large amount of dust, as well as there being a large number of stars and supernova remnants nearby.

  % see Dropbox/Research/Thesis/images/GalacticCenterInRadio.key for figure construction
  \begin{figure}[!t]
    \centering
    \includegraphics[width=0.90\textwidth]{images/GalacticCenterInRadio/GCIR.pdf}
    \caption[Galactic Center in Radio]{
      The center of our galaxy, viewed at a radio wavelength of $\lambda=90\text{cm}$.
      The dashed white line roughly represents the galactic plane, and the purple marks indicate the center of our galaxy.
      Supernova remnants and dust are visible in the view.
      The VERITAS field of view, the VERITAS point source spread (68\% containment), and the moon are shown for angular scale.
      Radio flux image is from Ref. \cite{galactic_center_in_radio}.
      \CaptionBlankLine
    }
    \label{fig_gc_radio}
  \end{figure}

  When there are multiple overlapping sources of gamma rays, it becomes difficult to discern which gamma rays came from which emission sources.
  The dust, supernova remnants, and the Galactic Center itself all emit gamma rays, making detection of a dark matter halo difficult.
  The dust operates as a collision target for high-energy protons from other galaxies, whose interactions produce $\pi_0$ particles, which then decay into gamma rays.
  These dust-induced gamma rays are collectively referred to as diffuse emission, and appear as a disk of gamma rays along the galactic plane.
  The supernova remnants accelerate electrons ($e^{-}$) and protons ($p$) outward, which then collide with the surrounding dust and gas forming a shockfront.
  This shockfront produces a spherical shell of gamma rays.
  Several of these shells (labeled with `SNR') can be seen in Figure~\ref{fig_gc_radio}, a radio-frequency image of the Galactic Center.
  Several other angular scales are also shown in this figure, including the moon and the VERITAS field of view.
  The black hole Sgr A* at the Galactic Center also produces gamma rays in a point-like shape (point-like relative to the other sources), though the mechanism by which it does this is not well understood~\cite{gal_cent_still_undetermined}.

  These effects all obscure the target of this analysis, the gamma ray emission from a dark matter halo.
  This spherically-symmetric halo of gamma rays would surround the Galactic Center, decreasing in intensity the further from the center.
  The halo's spectrum would also be different from the surrounding diffuse emission.

  Once these gamma rays have been emitted, they would travel to Earth, where humans can detect them.
  This is done by using a telescope to observe the particle shower that occurs when each gamma ray hits the Earth's atmosphere.
  The VERITAS telescope has been observing gamma rays from the Galactic Center since 2010.
  The data from these observations is used in this analysis.

\FloatBarrier

\section{Halo Analysis}
  After several years of operation, VERITAS has accumulated 108 hours of Galactic Center data, analyzed in Chapter~\ref{chapter:analysis}.
  To analyze these observations, an unbinned likelihood analysis is used, described in Section~\ref{sec:likeratio}.
  With this analysis, the Galactic Center region is modeled with nine different dark matter halos, where the magnitude of each halo is scaled to best fit the observations.
  For the spatial shape of the dark matter halo, a cuspy Einasto profile is chosen, as discussed in Section~\ref{dm_spatial}.
  For the spectral shape, only the $b\bar{b}$ annihilation channel is used, as discussed in Section~\ref{dm_spectral}.
  Each of the nine dark matter halos is modeled with a different dark matter mass $m_{\chi}$, which changes the spectrum of gamma rays produced in each $\chi\chi$ annihilation.
  
  From this analysis, Section~\ref{like_results} discusses how no dark matter signal was found at any of these masses.
  With this null result, the next step is to calculate the upper limit on the velocity-averaged cross section of this WIMP, discussed in Section~\ref{upper_limit}.
  From these upper limits, new limits can be placed on the WIMP velocity-averaged cross section at \SI{100}{\TeV{}}.


