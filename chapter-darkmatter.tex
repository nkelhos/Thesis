\cleartooddpage[\thispagestyle{empty}]
\chapter{Dark Matter Physics}

  As dark matter makes up 24\% of the universe's energy and 84\% of its mass \cite{wmap9year}, it has had a significant impact on the development of the universe and its present day distribution.
  % https://arxiv.org/pdf/1212.5226.pdf
  % table 3
  % Nine-year WMAP+BAO+H0b column
  % Omega_b = 0.0472 and Omega_c = 0.2408
  In this chapter, the main features of dark matter physics are discussed.
  This includes the astrophysical evidence for the existance of dark matter, an outline of the current cosmological paradigm of $\Lambda$CDM, as well as arguments for why dark matter may be in the form of a new, uknown particle.

\section{Astrophysical Evidence for Dark Matter}
  The current effects attributed to dark matter can be grouped into four different length scales.
  These effects also fall into one of three categories: gravity pulling on electromagnetic emitters, gravity bending background light, or by measuring the universe's total energy budget.
  On the smallest scales, concentrations of several thousand stars can be seen revolving around their center of mass, at a larger distribution of speeds than one would expect from the existing visible amount of matter.
  At larger scales the optical light from galaxies, as well as hydrogen lines, can be used to measure the amount of mass and its rotational velocity around the center of galaxies.
  At even larger scales, galaxy velocities can be measured and compared, x-ray telescopes can monitor the amount of hot gas, and mass-heavy areas of space will gravitationally lens background galaxies.
  At the the largest scale, the measurement of oscillations in the cosmic microwave background can be used to determine the amount of dark and baryonic matter.
  
  \subsection{$10^{19}\:\text{m}$ : Dwarf Galaxy Scale}
    % 10^19m comes from: 
    % fornax dwarf spheroidal galaxy wiki page
    %    17' x 12.6' in solid angle, call it 15'
    %    140kpc away  
    %    140kpc * Tan(15') = 0.6kpc = 1.8*10^19m ~ 10^19m
    At scales of $\nicetilde 10^{19}$m, groups of thousands stars, called satellite galaxies or dwarf galaxies, lie at the edge of full size galaxies.
    Telescopes can measure the individual spectra of these stars, allowing for their velocity to be calculated.
    Often, these observations focus on the Red Giant stars within these dwarf galaxies\cite{dwarf_gal_red_giant}.
    By looking at the distribution of these velocities (the width of this distribution is called the velocity dispersion), the total mass of the dwarf galaxy can be inferred\cite{dwarf_gal_vel_dispersion}, \cite{dwarf_gal_vel_dispersion2}.

    What makes this possible is that the velocity dispersion of a group of stars is proportional to the total mass of the graviational well.
    This can be seen by applying the virial theorem and maxwell-boltzmann statistics.

    Dwarf galaxies have mass-to-light ratios of around 5-10 $\frac{M_\odot}{L_\odot}$ {\color{red}(cite??)}, but can be as high as \nicetilde 1000 \cite{Simon2007_dwarfgalaxykeck}.
    
    In addition, it has been shown that dwarf galaxies near the Perseus cluster show no evidence of tidal disruption, while their baryonic mass alone would not prevent such disruption \cite{Penny2009}.

  \subsection{$10^{20}\:\text{m}$ : Galaxy Scale}
    %
    % galaxy rotation curve wiki page, M33 has curve measurements out to 50,000ly
    %   50,000ly = 4.7*10^20m ~ 10^20m
    At scales of $\nicetilde 10^{20}$m, the effects of Dark Matter on galaxies are observable.
    In one method, the total amount of light produced by a quadrant of a galaxy is measured with telescopes.
    Known mass-to-light ratios can then be used to calculate the total amount of mass within that quadrant.
    For example, in a survey of 25 galaxies in \cite{galaxy_mass_light_ratio}, most possesed a mass-to-light ratio of 2 to 6 {\color{red}(??)}.


    In a second method, a galaxy's emission spectrum is observed at many positions around its disk (center, outer edges, etc).
    By comparing the orientation of the disk with the doppler-shifted position of well-known spectral lines, one can calculate the average velocity that each chunk is travelling at around the center of the galaxy.
    Newton's law of gravity can then be used to calculate the mass contained within a sphere of that same radius.
    This calculation ends up with a larger amount of mass.

    {\color{red}Mass to light ratios??}

    Our own Milky Way galaxy is measured to have a mass to light ratio of 10 $\frac{M_{\odot}}{L_{\odot}}$.

    Galaxy rotation curves can also be measured through weak gravitational lensing \cite{weak_lensing_2001}.

  \subsection{$10^{23}\:\text{m}$ : Galaxy Cluster Scale}
    %
    % galaxy cluster wiki page
    % 2-10 Mpc, call it 6Mpc = 1.85*10^23m ~ 10^23m
    At scales of $\nicetilde 10^{23}$m, Dark Matter's effects on galactic clusters becomes observable by comparing three techniques.
    In the first technique, galaxy clusters are be massive enough to bend the images of background galaxies.
    The type and amount of bending can be used to determine the mass of the intermediate galaxy.

    X-Ray observations of galaxy clusters can measure the amount of hot baryonic mass.
    This has been shown distinctly in the Bullet Cluster\cite{bullet_cluster}.
    In this cluster, the total (baryonic+dark) mass was traced using gravitational lensing of background light.
    The baryonic mass was then traced with x-rays, which are emitted by ionized gas in the region.
    Figure \ref{fig:bullet} then shows these two masses overlayed.

    \begin{figure}[ht]
      \includegraphics[width=0.95\textwidth]{images/bulletcluster.eps}
      \caption[The Bullet Cluster]{
        The bullet cluster\cite{bullet_cluster_combined_image}.
        The blue clouds indicate the graviational lensing mass\cite{bullet_cluster}, the red represents two clouds of ionized baryons emitting x-rays\cite{bullet_cluster_chandramap}.
        The remaining stars and galaxies are imaged in the optical spectrum\cite{bullet_cluster_composite}.}
      \label{fig:bullet}
    \end{figure}

    Followup studies have also been done of other galaxy clusters.
    {\color{red}Mass to light ratios??}


    In the second method, the velocity of different galaxies in a cluster can be measured.
    By comparing the velocities of different galxies in a cluster, the contained mass within the cluster can be measured.
    {\color{red}Mass to light ratios??}

  \subsection{$10^{26}\:\text{m}$ : Universe Scale}
    %\subsection{Inter-Cluster Scale}
    % age of universe (13.82*10^9 years * speed of light) = 1.307*10^26m
    At the largest scale of the universe, $\nicetilde 10^{26}$m, the cosmic microwave background (CMB) has been used to measure the total amount of dark matter in the universe.
    By looking at the structure of the CMB, earlier times in the universe can be studied.
    Closer to the big bang, there was a point in time referred to as the 'freezout'.
    Before that time, the universe was a sea of quark plasma and other particles, and the temperature was too hot to allow baryons exist for any long period of time.
    After that freezout time, the temperature low enough that baryons stopped being created and annihilated.
    Thus, the number of baryons that were in the universe at the freezout temperature determines how many baryons are in existance today, since there aren't any significant ways to destroy universe-scale quantities of baryons.
    This freezout left an imprint of its structure, where more baryons in some places created more CMB photons, and other places had less of each.

    From fitting this structure, the composition of the universe can be fit.
    In terms of energy, 69.1\% of the universe is composed of Dark Energy, which is causing almost all galaxies in sight to accellerate away from the Milky Way.
    Another 4.9\% of the universe's energy is stored in baryonic matter, like protons and neutrons.
    The remaining 26\% of the universe's energy is contained in Dark Matter \cite{planck2015}.

    As another important piece of evidence is the abundance of Deuterium present in the universe.
    The amount of deuterium that was left over from the big bang depends on the amount of baryons in the universe.
    The amount of baryons predicted by this is not enough to account for the total mass of the universe, thus it must exist in other (darker) particles.
    Most theories of dark matter being locked up in cold, dead stars (MACHOs) are limited by this baryon limitation {\color{red}(cite??)}.

    Another measurement that depends heavily on the presence of dark matter is the rate at which galaxies form clumps.
    In the Sloan Digital Sky Survey, the positions of 1.6 million galaxies, quasars, and stars are mapped\cite{sdss_release}.
    By simulating the distribution of similar objects as the universe ages, only with a dark matter component does the universe form clumps that match SDSS observations.
  
    A much more detailed discussion on the evidence for dark matter can be read here \cite{DMPrimer}.


\section{$\Lambda$CDM Cosmology}

  At the largest spatial scales, $\Lambda$CDM is the currently accepted model of cosmology.
  The $\Lambda$ refers to the density of dark energy, while CDM refers to Cold Dark Matter.
  $\Lambda$CDM models how the universe expanded from the big bang until the present day universe, using .
  The predictions this model makes revolve around how different particles condensed out of the universe at different times, and how their distribution throughout the universe changed over time.

  A large part of $\Lambda$CDM comes from measuring fluctuations in the Cosmic Microwave Background (CMB).
  In these fluctuations, a snapshot of the universe when it was 380,000 years old is preserved, providing many clues as to the development of the universe, including its age, energy content, structure, and expansion. 
  This CMB snapshot also provides hints about the nature of dark matter, including some basic requirements for any potential dark matter candidate particle.

  \begin{figure}[ht]
    \includegraphics[width=0.95\textwidth]{images/cmb_skymap/cmb_skymap.eps}
    \caption[The Cosmic Microwave Background]{
      The cosmic microwave background temperature map of the universe \cite{wmap_skymap}, from 9 years of WMAP observations \cite{wmap9year}.
      This image shows a temperature range of \SI{\pm200}{\mu{}Kelvin}.
    }
    \label{fig:cmb}
  \end{figure}

\section{Particle Dark Matter}

  The current paradigm of particle physics is called the Standard Model \cite{standardmodel} {\color{red}(This is a random book on SM, is there a better one??)}.
  It consists of groups of particles called quarks and leptons, as well as the bosons that mediate interactions between these particles.
  Quarks combine to form hadrons, like protons and neutrons, and mesons, while leptons consist of electrons, muons, tauons, and their neutrino companions.
  Since all Standard Model particles have been eliminated as Dark Matter candidates {\color{red}(cite??)}, theoretically predicted particles are now the focus of many searches.
  At the forefront of these searches are particles predicted by Supersymmetry \cite{Jungman:1995df}, (specifically Minimal Supersymmetric Standard Model, or MSSM \cite{MSSM}), an extension to the Standard model.
  Much like many particles have an anti-particle, in supersymmetry, each Standard Model quark, lepton, or boson, has a supersymmetric partner particle.
  Quarks and leptons have squarks and sleptons as their supersymmetric partners, while bosons have partners like photinos, gluinos, and charginos.
  While no definitive evidence of supersymmetry has been discovered to date{\color{red}(??)}, it is still preferred due to its ability to predict physics across a large range energy scales, the holy grail of any Grand Unified Theory.

  {\color{red}(why neutralino??)}
  
  While the currently favored dark matter candidate is the neutralino, the lightest supersymmetric particle, a more general term for dark matter candidate particles that meets the conditions required by cosmology and particle physics is a WIMP, or Weakly Interacting Massive Particle.
  One of the major predictions from particle physics and cosmology is referred to as the "WIMP Miracle".
  In it, particle physics and cosmology separately predict that if dark matter is WIMP, it should have a velocity-averaged cross section of around $3*10^{-27}\text{cm}^{3}\text{s}^{-1}$, though each field comes to this value by very separate math. {\color{red}(cite??)}

  In the 1990s, supersymmetry predicted the existance of a WIMP-like particle with a crosssection of around \nicetilde \SI{3e-26}{cm$^2$} \cite{Jungman:1995df}.
  % $3*{10}^{-26}\ {\textrm{cm}}^{2}$
  What also made this WIMP particle a promising candidate is that it also solves several problems in cosmology, and that such a WIMP-like particle would have a similar cross section.
  Thus, two separate fields of physics came together and noticed that they both were looking for a WIMP-like particle within a mass and crosssection range.
  In $\Delta$CDM cosmology, the relic density is all of the leftover dark matter particles after the universe became too cold to produce more dark matter particles.

  {\color{red} relic density??}

  There are other potential dark matter candidates, both as other particle types or as modifications to other areas of physics, but they are not explored in this thesis.
  See \cite{DMPrimer} for more a more detailed discussion.



