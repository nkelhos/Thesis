\cleartooddpage[\thispagestyle{empty}]
\chapter{Dark Matter}

\section{Evidence for Dark Matter}

Peering out into space, there is evidence for dark matter on several different length scales.
As this dark matter, by definition does not interact with electromagetic light in any significant way, most of the evidence for dark matter relies on a combination of gravtiational effects pulling on electromagnetic emitters, or mass distorting the paths taken by the light.

On the smallest scales, groups of several thousand stars can be seen revolving around a center of mass, at a larger spread of speeds than allowed by the existing visible amount of matter.
At larger scales the optical light from galaxies, as well as hydrogen lines, can be used to measure the amount of mass and its rotational velocity around the center of galaxies.
At even larger scales, galaxy velocities can be measured and compared, x-ray telescopes can monitor the amount of hot gas, and mass-heavy areas of space will gravitationally lens background galaxies.
At the the largest scales, the measurement of oscillations in the dipole spectrum of the cosmic microwave background can be used to determine the amount of dark matter.


\subsection{Dwarf Galaxy Scale}
% 10^19m comes from: 
% fornax dwarf spheroidal galaxy wiki page
%    17' x 12.6' in solid angle, call it 15'
%    140kpc away  
%    140kpc * Tan(15') = 0.6kpc = 1.8*10^19m ~ 10^19m
At scales of $\nicetilde 10^{19}$m, groups of several thousand stars, called satellite galaxies, lie at the edge of the Milky Way galaxy.
Telescopes can measure the spectra of these stars, allowing for their velocity to be calculated.
By looking at how fast each star is moving around the satellite galaxy's center of mass, and how widely these rotational velocities differ, the amount of enclosed mass can be inferred.

\subsection{Galaxy Scale}
%
% galaxy rotation curve wiki page, M33 has curve measurements out to 50,000ly
%   50,000ly = 4.7*10^20m ~ 10^20m
At galactic scales of $\nicetilde 10^{20}$m, the effects of Dark Matter on galaxies is observable.
The total amount of light produced by a quadrant of a galaxy is measured with telescopes.
Known mass-to-light ratios can then be used to calculate the total amount of mass within that quadrant.
This was the earliest method of measuring the mass of a galaxy, producing a low calculation for the amount of mass.

In the third method, a galaxy's emission spectrum is observed at many positions around its structure (center, outer edges, etc).
By comparing the orientation of the disk with the doppler-shifted position of common spectral lines, one can calculate the average velocity that each chunk is travelling at around the center of the galaxy.
Newton's law of gravity can then be used to calculate the mass contained within a sphere of that same radius.
This calculation ends up with a larger amount of mass.

\subsection{Galaxy Cluster Scale}
%
% galaxy cluster wiki page
% 2-10 Mpc, call it 6Mpc = 1.85*10^23m ~ 10^23m
At scales of $\nicetilde 10^{23}$m, Dark Matter's effects on galactic clusters becomes observable.

\subsubsection{Gravitational Lensing}
The large masses of galaxy clusters can be enough to bend the images of background galaxies, focusing them onto observatories on Earth.
The type and amount of bending can be used to determine the mass of the intermediate galaxy.

\subsubsection{Galaxy Group Velocities}

By studying the velocity of different galaxies in a cluster, the contained mass within the cluster can be measured.

\subsubsection{X-Ray Observations}
X-Ray observations of galaxy clusters can measure the amount of hot baryonic mass.


In the fourth method, a large swath of space can be observed with high-resolution telescopes.
Volumes of space with large amounts of matter will bend the light from background galaxies in known ways.
By examining how the images of different background galaxies are bent, the amount of mass between the galaxy and the observer can be inferred.
This is typically used for measuring the mass of cluster-sized volumes of space, rather than individual galaxies.
xray gas??

\subsection{Universe Scale}
%
% age of universe (13.82*10^9 years * speed of light) = 1.307*10^26m
At the largest scale of the universe, $\nicetilde 10^{26}$m, the cosmic microwave background (CMB) has been used to measure the total amount of dark matter in the universe.
By looking at the structure of the CMB, earlier times in the universe can be studied.
Closer to the big bang, there was a point in time referred to as the 'freezout'.
Before that time, the universe was a sea of quark plasma and other particles, and the temperature was too hot to allow baryons exist for any long period of time.
After that freezout time, the temperature had dropped to the point where more baryons were created than destroyed, much like cooling steam condensing to liquid water.
This freezout left an imprint of its structure, more CMB photons in some places, less in others.
By looking at this CMB structure, the total number of baryons in the universe can be measured.
??Volume measured??


\section{The Standard Model}

The Standard Model of particle physics describes the current paradigm of particles, including their masses, charges, and crosssections.
Most particles in the universe are made from combinations of quarks and leptons, with bosons mediating the forces.

The six quarks, up, down, charm, strange, top, and bottom, along with their anti-quarks, can combine in twos and threes (and fives??) to become composite baryon particles.
Protons and neutrons are the most common in the universe, consisting of up-up-down and up-down-down quarks, respectively.
These composite particles are bound together via the strong force, acting on each quark's color charge, but also carry electric charge.

Six leptons then make up the remaining massive particles, in the form of the electron $e$, the muon $\mu$, and the tauon$\tau$ ??, and their neutrino versions $\nu_e$, $\nu_\mu$, and $\nu_\tau$.

To transmit forces and provide mass, 5 bosons are known, including the photon to mediate the electromagnetic force, the W and Z bosons to mediate the weak force, the gluon to mediate the strong force, and the Higgs boson that provides mass??.

\subsection{Dark Matter Particle Candidates}

There are several candidate particles that may be dark matter.

WIMPs
Axions
???

\section{Particle Accelerators}

Supernova remnants, Pulsar Wind Nebulae, any kind of jet


\section{Search Techniques}

% direct
% collider
% indirect

There are three general methods of search for dark matter particles.
Direct Detection experiments attempt to discern when a dark matter particle directly strikes a volume of detection mass, stored deep underground.
Indirect Detection experiments attempt to look for secondary products of dark matter interactions, usually through astrophysical searchs.
Collider searches attempt to produce a dark matter particle through controlled collisions of standard model particles.

Direct searches consist of an interaction mass, surrounded by light, heat, and acoustic sensors, buried deep in mineshafts.
These sensors attempt to detect when a particle has struck the enclosed interaction mass, and discern how much mass and energy the incident particle had.
Most of these detectors are either solid blocks of ??, or liquified gasses like ??.

Collider searches involve accelerating particles in particle accelerators, smashing them together, and searching the resulting explosion of particles for evidence of dark matter particles.

Indirect searches involve observing regions of space with telescopes, looking for secondary particles that may have been produced by a dark matter collisions.
This thesis constitutes an indirect search, examining whether gamma rays near the galactic center match the given predicted shape of a dark matter halo.
Much of the evidence for dark matter, as detailed in section ??, comes from indirect searches.



\section{Mass and Cross-section}


\section{Old}

Dark matter was named as a solution to an inconsitancy in galactic rotation curves.

Hydrogen spectral lines were studied from various places in different galaxies.
From these spectral lines, doppler shifts could be used to calculate the radial velocity of that area of the galaxy.
By comparing the radial velocities at different distances from the visible center of galaxies, a profile can be created that charts the enclosed mass.

Alternately, star formation can be simulated, which can be used to predict the amount of light seen for each solar mass in a region.
From this second method, the mass contained in a galaxy can also be measured at each radial distance from the galaxy's center.

By comparing these two mass curves, it has been found that they differ greatly.
Thus, the concept of dark matter was invented to make up for the discrepancy between these two curves.

% core profile

% einasto


Dark Matter is viewed as the seed for where galaxy clusters form in the universe.


Due to how the universe formed, the missing matter is not percieved to be locked up in baryons.

% discuss how cosmology rules out
