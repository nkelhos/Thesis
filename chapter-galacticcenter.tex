\cleartooddpage[\thispagestyle{empty}]
\chapter{The Galactic Center}

The galactic center is a complex region of space, with many astrophysical sources of gamma rays.
A disk of dust lies along the galactic plane, acting as an interaction medium for diffuse proton cosmic rays.
Nearby supernova remnants also produce gamma-rays as their expanding shell interacts with ambient dust.
The immediate area surrounding the galactic center a point-source emitter of gamma rays, though this mechanism is uncertain.

\section{Supermassive Black Hole}

Through kinematic observations of nearby stars, the galactic center is suspected to have a black hole, on the order of $10^6 M_{\odot}$. ??

\section{Diffuse Emission}
The disk of gas that permeates the galactic plane acts as an interaction medium for passing cosmic rays, both from galactic accelerators, as well as from extragalactic sources.
These high-energy protons collide with the dust, shattering into $\pi^{\pm,0}$.
The $\pi^0$ then decays into two gamma rays ??, providing the diffuse emission.
The spectrum of this emission is ??.

\section{Nearby Astrophysical Sources}

\subsection{Point Sources}

\subsection{Supernova}

