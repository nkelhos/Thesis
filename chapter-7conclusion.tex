\cleartooddpage[\thispagestyle{empty}]
\newcommand{\nicetilde}{{\raise.17ex\hbox{$\scriptstyle\mathtt{\sim}$}}}
\chapter{Conclusion}

A likelihood ratio test was used with VERITAS data to measure the gamma-ray emission from the Galactic Center.
This test searched for a halo of WIMP dark matter particles that annihilated into $b\bar{b}$ quarks, which then cascaded into other particles, including some gamma rays.
The halo model utilized in this analysis is a simple Einasto halo profile, though without any core flattening.
Nine different WIMP masses were tested in this search.
The VERITAS instrument response functions were also folded into the WIMP halo models, to account for dispersion in the reconstructed gamma ray position and energies.
This particular analysis is unique in that observations were taken at an extremely low (\nicetilde{}\ang{30}) telescope elevation.
This reduced the overall sensitivity, but also shifted the sensitive energy range upwards to \SIrange{1}{100}{\TeV{}}.
Camera background models in this analysis were made using several hours of dedicated dark observations, of a region of the sky where there are no other gamma-ray emitters.
No significant Dark Matter signal was detected at any of the WIMP masses.

To illuminate what WIMP candidates were ruled out by this search, a cross section upper limit was calculated for each of the nine WIMP masses.
Due to the VERITAS sensitivity to TeV gamma rays at low elevations, new limits were placed at WIMP masses above \SI{70}{\TeV{}}.
This likelihood ratio test provides improved sensitivity over the standard VERITAS On-Off analysis method, granting stronger limits with fewer hours of observation.
The ratio test described here can also model other source types, providing this sensitivity boost to all VERITAS observation targets.

When a residual skymap is made of the differences between the best-fit models and the data, a strong gradient is noted along the observatory's elevation axis.
This gradient is due to the camera background models not properly handling the atmospheric gradient, which at \ang{30} elevation is \nicetilde{}20\% different between the top and bottom of the VERITAS field of view.
By adding a simple elevation gradient to the background models, the sensitivity of the Dark Matter upper limit to the background modeling was tested.
It was found that for a moderate gradient of 5\%/degree, the upper limit at one $m_{\chi}$ only moved by 0.1\%, indicating it is not strongly dependent on the background shape.

There are several improvements which can be made when using this analysis method in future VERITAS studies.
The background models created for this thesis indicate that at lower energies the background shape is more complex than previously thought.
These background models can be improved to allow for the inclusion of low energy events from \SIrange{1.5}{4}{\TeV{}}.
This additional energy range would add \nicetilde{}40\% more events to the analysis, improving Dark Matter searches and upper limits.
However, the causes of the variations in camera sensitivity are not well understood yet, and need to be explored further.
These variations are at least partly due to the atmospheric gradient.
Since gamma-ray showers and proton showers behave similarly in the atmosphere, it may also mean that the gamma-ray effective areas and energy dispersions also depend on the (x,y) position within the camera, instead of just the camera radius.
Using diffuse gamma-ray simulations may offer improvements to the effective area and energy dispersion for all VERTAS analyses.
More studies would need to be done to explore this.

Another benefit of this work is that future VERITAS analyses can be combined with other IACTs like H.E.S.S., MAGIC, and CTA.
This would allow for more detailed temporal studies, studies across larger energy ranges, and increased analysis sensitivity, due to the large number of observing hours that would be combined.


