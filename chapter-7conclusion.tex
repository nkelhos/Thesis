\cleartooddpage[\thispagestyle{empty}]
\newcommand{\nicetilde}{{\raise.17ex\hbox{$\scriptstyle\mathtt{\sim}$}}}
\chapter{Conclusion}

A likelihood ratio test was used with VERITAS data to analyze the gamma-ray emission from the Galactic Center.
This test searched for a halo of WIMP dark matter particles that annihilated into $b\bar{b}$ quarks, which then cascaded into other particles, producing some gamma rays in the process.

The halo model utilized in this analysis is a simple Einasto halo profile, though without any core flattening.
Nine different WIMP masses were tested in this search.
The VERITAS instrument response functions were also folded into the WIMP halo models, to account for dispersion in the reconstructed gamma ray position and energies.
This particular analysis is unique in that observations were taken at an extremely low (\nicetilde{}\ang{29}) telescope elevation.
This reduced the overall sensitivity, but also shifted the sensitive energy range upwards to \SIrange{4}{70}{\TeV{}}.
Camera background models were made using several hours of dedicated dark observations, of a region of the sky where there are no other gamma-ray emitters.

During the analysis, a strong gradient is noted in the background models along the observatory's elevation axis.
This gradient occurs because of the low telescope elevation, at which the atmospheric air mass varies across the field of view.
By accounting for this gradient, the upper limit at one $m_{\chi}$ only moved by 0.1\%, indicating it does not strongly depend on the atmospheric air mass.

The result of this analysis is that no significant dark matter signal was detected at any of the WIMP masses.
To illuminate what WIMP candidates were ruled out by this search, a cross section upper limit at a confidence level of 95\% was calculated for each of the nine WIMP masses.
Due to the VERITAS sensitivity to TeV gamma rays at low elevations, new limits were placed at high WIMP masses.
For $70\,\TeV<m_{\chi}<100\,\TeV$, these upper limits exclude WIMP$_{\chi\chi\rightarrow\Pbottom\APbottom}$ $\left \langle \sigma v \right \rangle$'s above \SI{7.627e-25}{cm^3/s} at the 95\% confidence level.
This results in an improvement over the previous Fermi-MAGIC limit by more than an order of magnitude above \SI{70}{\TeV{}}.
{\color{red}(how do your limits compare to other limits? describe a bit more your results?? -orel)}

This work can be expanded in several different ways.
In addition to searching for different dark matter halos and annihilation spectra, the background modeling can be improved to include low-energy events, to increase the sensitivity.
This may allow for new limits at a larger range of masses.
Sensitivity could be further increased by including ~\SI{800}{hours} of VERITAS observations of several dwarf-spheroidal galaxies.
The VERITAS data used here can also be combined with other IACTs like H.E.S.S, MAGIC, and CTA, to perform more sensitive searches for dark matter.
{\color{red}(how does this outlook paragraph sound??)}

