\cleartooddpage[\thispagestyle{empty}]
\newcommand{\nicetilde}{{\raise.17ex\hbox{$\scriptstyle\mathtt{\sim}$}}}
\chapter{Conclusion}


A likelihood ratio test was used with VERITAS data to analyze the gamma-ray emission from the Galactic Center.
This test searched for a halo of WIMP dark matter particles that annihilated into $b\bar{b}$ quarks, which then cascaded into other particles, producing some gamma rays in the process.

The halo model utilized in this analysis is a simple Einasto halo profile, though without any core flattening.
Nine different WIMP masses were tested in this search.
The VERITAS instrument response functions were also folded into the WIMP halo models, to account for dispersion in the reconstructed gamma ray position and energies.
This particular analysis is unique in that observations were taken at an extremely low (\nicetilde{}\ang{29}) telescope elevation.
This reduced the overall sensitivity, but also shifted the sensitive energy range upwards to \SIrange{4}{70}{\TeV{}}.
Camera background models were made using several hours of dedicated dark observations, of a region of the sky where there are no other gamma-ray emitters.

During the analysis, a strong gradient is noted in the background models along the observatory's elevation axis.
This gradient occurs because of the low telescope elevation, at which the atmospheric air mass varies across the field of view.
By accounting for this gradient, the upper limit at one $m_{\chi}$ only moved by 0.1\%, indicating it does not strongly dependend on the atmospheric air mass.

There are several improvements which can be made when using this analysis method in future VERITAS studies.
VERITAS analyses typically use radial backgrounds, as no other dependence aside from camera $x$/$y$ and energy had been demonstrated until now.
The background studies in this thesis indicate that at lower energies the background shape also depends on the atmospheric column density.
The background models created for this thesis can be further improved to allow for the inclusion of low energy events, which would improve the sensitivity of future dark matter searches.
However, the causes of the variations in camera sensitivity are not well understood yet.
These variations are at least partly due to the atmospheric gradient, though more studies are needed to explore this.

An extra benefit of adapting VERITAS data to the CTOOLs format is that future VERITAS analyses can be combined with other IACTs like H.E.S.S., MAGIC, and CTA.
{\color{red}(Why is this a benefit of this work? (is this better??) -orel)}
This would allow for more detailed temporal studies, studies across larger energy ranges, and increased analysis sensitivity, due to the large number of observing hours that would be combined.

The result of this analysis is that no significant dark matter signal was detected at any of the WIMP masses.
To illuminate what WIMP candidates were ruled out by this search, a cross section upper limit at a confidence level of 95\% was calculated for each of the nine WIMP masses.
Due to the VERITAS sensitivity to TeV gamma rays at low elevations, new limits were placed at high WIMP masses.
For $70\,\TeV<m_{\chi}<100\,\TeV$, these upper limits eliminate WIMP$_{\chi\chi\rightarrow\Pbottom\APbottom}$ $\left \langle \sigma v \right \rangle$'s above \SI{7.627e-25}{cm^3/s}.
{\color{red}(how do your limits compare to other limits? describe a bit more your results?? -orel)}
This unbinned likelihood ratio test provides improved sensitivity over the standard VERITAS On/Off analysis method, granting stronger limits with fewer hours of observation {\color{red}(Li & Ma is also derived from a likelihood ratio test. So I don’t think this can be used as the reason for a sensitivity boost. Please correct this paragraph?? -gernot)}.
The ratio test described here can also model other source types, providing this sensitivity boost to all VERITAS observation targets.
{\color{red}(there are a bit too many details in this paragraph?? -orel)}

{\color{red}(I think the order is still somewhat problematic.
You talk about the outlook before your results.
Describe the analysis, give the result, discuss its ramifications and at the end if you wish, give an outlook on what can be improved or how methods from the analysis can be relevant for other studies ?? -orel)}
