\cleartooddpage[\thispagestyle{empty}]
\newcommand{\nicetilde}{{\raise.17ex\hbox{$\scriptstyle\mathtt{\sim}$}}}
\chapter{Conclusion}

A likelihood ratio test was used with 108 hours of VERITAS data to analyze the gamma-ray emission from the Galactic Center.
This test searched for a halo of WIMP dark matter particles that annihilated into $b\bar{b}$ quarks, which then cascaded into other particles, producing some gamma rays in the process.
Observations were taken at an extremely low (\nicetilde{}\ang{29}) telescope elevation.
This reduced the overall sensitivity, but also shifted the sensitive energy range upwards to \SIrange{4}{70}{\TeV{}}.

The halo model utilized in this analysis is a simple Einasto halo profile, without any core flattening.
Annihilation spectra were generated for WIMPs in the \SIrange{4}{100}{\TeV{}} mass range.
The VERITAS instrument response functions were folded into the WIMP halo models, to account for dispersion in the reconstructed gamma ray position and energies.
Camera background models were made using several hours of dedicated dark observations, of a region of the sky where there are no gamma-ray emitters.
A low-elevation atmospheric gradient was noted in the background models, however this was found to have a negligble effect on the calculated upper limits.

The result of this analysis is that no significant dark matter signal was detected in the mass range \SIrange{4}{100}{\TeV{}}.
To illuminate what WIMP candidates were ruled out by this search, cross section upper limits were calculated.
Due to the VERITAS sensitivity to TeV gamma rays at low elevations, new limits were placed at very high masses.
For $70\,\TeV<m_{\chi}<100\,\TeV$, these upper limits exclude WIMP$_{\chi\chi\rightarrow\Pbottom\APbottom}$ $\left \langle \sigma v \right \rangle$'s above \SI{7.627e-25}{\frac{cm^3}{s}} at the 95\% confidence level.
This results in an improvement over the previous Fermi-MAGIC limit by more than an order of magnitude.

This work can be expanded in several different ways.
In addition to searching for different dark matter halos and annihilation spectra, the background modeling can be improved to include low-energy events, to increase the sensitivity.
This may allow for new limits at a larger range of masses.
Sensitivity to dark matter could be further increased by including \nicetilde{}800 hours of VERITAS observations of several dwarf spheroidal galaxies.
The VERITAS data used here can also be combined with other IACTs like H.E.S.S, MAGIC, and CTA, to perform more sensitive searches for dark matter.

