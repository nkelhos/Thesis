\cleartooddpage[\thispagestyle{empty}]
\chapter{Gamma Ray Reconstruction}

VERITAS is a particle physics observatory.
Since it detects gamma rays by imaging their air shower, software is needed to reconstruct the shower.
This reconstruction recovers the point-of-origin of the gamma ray, as well as its energy, and the effective area with which it was detected.
This reconstruction process goes through several stages.
First, the images are identified, and the intersection point of the axes' images can be used to identify the original sky direction of the gamma ray.
Then, simulations are consulted to lookup which energy and effective area best fit each particular gamma ray.


\section{Pixel Selection}
Raw data from the telescope is saved to a file.
This data consists of the number of DC counts from each PMT at roughly nanosecond intervals.
When the telescope triggers on a gamma ray, however, it saves several nanoseconds forwards and backwards from the time of the trigger.

\section{Image Selection}


\section{Sky Direction}
image cleaning
hillas fitting
image intersection

\subsection{DISP Technique}


\section{Energy Reconstruction}
table lookup via shower size and distance

\section{Effective Area}

simulate events
spawn events in area
measure number that fall within the area

\section{Acceptance}
radial acceptance, fill gamma-like events to histogram

\section{Background}
fill gamma-like events to 2d histogram

\section{Point Spread Function}
spawn events at a single point, look at reconstructed distribution

\section{Energy Dispersion}
spawn events at single energy, look at reconstructed distribution

\section{BDT Method}
supervised learning algorithm for sorting gammas and protons, good for extended sources

