\cleartooddpage[\thispagestyle{empty}]
\phantomsection
\section*{Abstract}
\addcontentsline{toc}{chapter}{Abstract (Deutsche)}

Dunkle Materie macht 24\% der Energie des Universums aus, aber die Form, in der sie gespeichert wird, ist derzeit nicht bekannt.
Zu verstehen, welche Form diese Materie hat, ist eines der größten ungelösten Rätsel der modernen Physik.
Für dunkle Materie gibt es viele Hinweise auf Messungen von Galaxien, Zwerggalaxien, Galaxienhaufen und kosmologischen Messungen.
Eine Theorie besagt, dass dunkle Materie ein neues, unentdecktes Teilchen ist, das nur über die Schwerkraft und die schwache Kraft interagiert, die als \textit{schwach wechselwirkendes massives Teilchen} (WIMP) bezeichnet wird.
Ein WIMP-Kandidat ist ein supersymmetrisches Teilchen, das als Neutralino bezeichnet wird.
Das Ziel dieser Arbeit ist es, nach diesen Partikeln der dunklen Materie zu suchen und zu versuchen, ihre Masse und ihren Querschnitt zu messen.

Partikel der dunklen Materie scheinen sich in den meisten Gravitationsbohrungen im Galaxienbereich zu konzentrieren.
Eine Region des Weltraums, die sich in der Nähe befindet und eine hohe Dichte dunkler Materie besitzt, ist das Zentrum unserer eigenen Galaxie.
Es wird erwartet, dass das Neutralino zu Teilchen des Standardmodells vernichtet wird, die in Photonen zerfallen können.
Daher kann eine Suche nach Gammastrahlen in der Nähe des Galaktischen Zentrums die Anwesenheit von dunkler Materie aufdecken.

108 Stunden VERITAS-Gammastrahlenbeobachtungen des Galaktischen Zentrums werden in einer Analyse für ungebundene Likelihood-Analysen zur Suche nach dunkler Materie verwendet.
Die geringe Höhe des Galaktischen Zentrums führt dazu, dass VERITAS Gammastrahlen im Energiebereich \SIrange{4}{70}{\TeV{}} beobachtet.
Die in dieser Arbeit verwendete Analyse besteht aus der Modellierung des Halos dunkler Materie im Galaktischen Zentrum sowie des Spektrums der Gammastrahlen, die erzeugt werden, wenn zwei WIMPs vernichten.
Eine Punktquelle wird hinzugefügt, um die vom Galactic Center detektierte Gammastrahlenemission von nicht dunkler Materie zu modellieren.
Hintergrundmodelle werden aus Daten separater Beobachtungen außerhalb des Galactic-Centers erstellt.

Im Massenbereich \SIrange{4}{100}{\TeV} wird kein Signal der dunklen Materie gefunden.
Es wurden obere Grenzwerte für den durch die Geschwindigkeit gemittelten Querschnitt der WIMP berechnet, die oberhalb von \SI{70}{\TeV{}} zu neuen Grenzwerten von ${ \vacs{} < \left (6.6-7.6 \right ) \times 10^{-25} \frac{\mathrm{cm}^3}{\mathrm{s}} }$ auf dem Konfidenzniveau von 95\%.


%% add a blank page for margin styling
%\newpage
%\null
%\thispagestyle{empty}
%\newpage

\cleartoevenpage[\thispagestyle{plain}]
\null
