\cleartooddpage[\thispagestyle{empty}]
\phantomsection
\section*{Abstract}
\addcontentsline{toc}{chapter}{Abstract (Deutsche)}

VERITAS ist ein Gamma-Ray-Observatorium, das seit einigen Jahren astrophysikalische Gamma-Ray-Ziele wie das Galactic Center beobachtet.
Da VERITAS sich in der nördlichen Hemisphäre befindet, bedeutet dies, dass das Teleskop Gammastrahlen in geringer Höhe erfasst.
Dies führt dazu, dass das Teleskop für Gammastrahlen mit höherer Energie empfindlicher ist, als wenn es direkt darüber gerichtet wäre.
Da sich das Galaktische Zentrum in geringer Höhe befindet und ein vorrangiges Ziel für die Suche nach Dunkler Materie ist, ist eine Untersuchung der Dunklen Materie auf unerforschte Energien möglich.
In dieser Dissertation werden die Analysemethoden vorgestellt, die zur Suche nach Dunkler Materie in der Nähe des Galaktischen Zentrums aus Gammastrahlen mit Energien zwischen 4 und 70 TeV verwendet werden.


%% add a blank page for margin styling
%\newpage
%\null
%\thispagestyle{empty}
%\newpage

\cleartoevenpage[\thispagestyle{plain}]
\null
