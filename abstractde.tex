\cleartooddpage[\thispagestyle{empty}]
\phantomsection
\section*{Abstract}
\addcontentsline{toc}{chapter}{Abstract (Deutsche)}


Dunkle Materie bindet etwa 24 \% der gesamten Energie im Universum.
Bis heute ist jedoch dessen Ursprung nicht bekannt.
Dies ist eines der gr{\"o}{\ss}ten ungel{\"o}sten  R{\"a}tsel in der modernen Physik.
Untersuchungen von Galaxien, Zwerggalaxien, Galaxienhaufen und kosmologischen Messungen deuten auf Dunkle Materie hin.
Ein Kandidat f{\"u}r Dunkle Materie ist das sogenannte Weakly Interactive Massive Particle (WIMP), welches nur der Schwerkraft und der schwachen Wechselwirkung unterliegt.
Eines dieser supersymmetrischen Teilchen ist das Neutralino.
Das Ziel dieser Arbeit ist es, nach Dunkler Materie in dieser Form zu suchen und deren Eigenschaften, wie Masse und Wechselwirkungsquerschnitt, zu untersuchen.

Dunkle Materie verdichtet sich im Bereich von starken Gravitationsfeldern.
%Aufgrund seiner N{\"a}he sowie der hohen Sternendichte bietet das Zentrum unserer Galaxie besondere M{\"o}glichkeiten zur Suche nach Dunkler Materie.
Aufgrund seiner N{\"a}he sowie der hohen Dichte an Dunkler Materie bietet das Zentrum unserer Galaxie besondere M{\"o}glichkeiten zur Suche nach diesen Teilchen.
Es wird vermutet, dass Neutralinos miteinander wechselwirken, dabei in Teilchen des Standard Modells zerfallen und so Photonen mit hohen Energien entstehen.
Die Suche nach hochenergetischen Gammastrahlen in der N{\"a}he des Galaktischen Zentrums kann folglich das R{\"a}tsel der Dunklen Materie l{\"o}sen.

Das Gammastrahlenobservatorium VERITAS hat das Galaktische Zentrum f{\"u}r etwa 108 Stunden beobachtet.
Diese Daten wurden mittels einer unbinned Likelihood-Analyse auf die Existenz von Dunkler Materie untersucht.
Da VERITAS das Galaktische Zentrum bei geringer Elevation beobachtet, k{\"o}nnen nur Gammastrahlen in einem Energiebereich zwischen 4 und 70 TeV detektiert werden.
Die in dieser Arbeit benutzte Analysemethode besteht zum einen aus der Modellierung des Galaktischen Zentrums inklusive dessen Dunkle-Materie-Halo.
Zum anderen wird das Gammastrahlenspektrum, welches bei der Wechselwirkung zweier WIMP-Teilchen entsteht, untersucht.
Der Beitrag der Gammastrahlen, welcher nicht von Dunkler Materie erzeugt wird, ist mittels einer punktf{\"o}rmigen Quelle modelliert.
Zum Schluss wird der Untergrund mit realen Daten au{\ss}erhalb des Galaktischen Zentrums abgesch{\"a}tzt.

Im Energiebereich zwischen 4 und 100 TeV wurde keine Signale der Dunklen Materie gefunden.
Obere Grenzwerte f{\"u}r den Wechselwirkungsquerschnitt der WIMPs ergeben ${\vacs < (6.6-7.6) \times 10^{-25} \frac{\mathrm{cm}^3}{\mathrm{s}}}$ oberhalb von 70 TeV in einem 95-prozentigen Erwartungsintervall.




%% add a blank page for margin styling
%\newpage
%\null
%\thispagestyle{empty}
%\newpage

\cleartoevenpage[\thispagestyle{plain}]
\null
