\cleartooddpage[\thispagestyle{empty}]
\chapter{Gamma Rays}


\section{Methods of Creation}

\subsection{Leptonic Processes}
Electrons up-scatter ambient photons into higher-energy gamma rays.
Need an electron accelerator, and need photons of a necessary spectrum.
Spinning pulsars, when their magnetic field lines break and reconnect, can accelerate electrons. (??)

\subsection{Hadronic Processes}
High energy protons collide with other protons, decay into pions, and neutral pions ($\pi_{0}$) decay into gamma-ray pairs.
These protons can be 

Electromagnetic showers are formed by high energy electrons or gamma-rays, and consist primarily of electrons, positrons, and photons.
Through successive particle generations in the shower, photons will interact with ambient atmospheric atoms to convert into electron-positron pairs.
Electrons and positrons will produce bremstrahlung photons as they suffer accelerations, and positrons interact with ambient electrons to annihilate into photon pairs.
Eventually the photons don't posess enough energy to produce new $e^{+}e^{-}$ pairs, electrons quickly lose energy to mechanisms other than bremstrahlung, and the shower dies out.


\subsection{Dark Matter Interactions}
Indirect Detection: dark-dark => sm-sm
Direct Detection  : dark-sm => dark-sm
Collider Detection: sm-sm => dark-dark

Indirect can produce gamma rays via quark/lepton annihilation (??)
dark-dark => t-t => gamma-gamma
dark-dark => b-b => gamma-gamma


\section{Atmospheric Shower}

Gamma rays strike atmospheric molecules, and produce an $e^\pm$ pair.
The $e^+$ quickly annihilates into a $\gamma$ pair, which then repeate the process.
the $e^-$ travels futher, collides with an existing particle, knocking more electrons free from atmospheric molecules.
These electrons (still posessing a large amount of energy) then go on to knock more electrons free, repeating the cycle.

As these electrons pass through the atmosphere, the are traveling at some velocity $c_{atmo}<v_e<c_{vac}$.
This means they produce cherenkov light, at an angle proportional to their velocity and the local speed of light.


\section{Galactic Center}

The galactic center is a complex region of space, with many astrophysical sources of gamma rays.
A disk of dust lies along the galactic plane, acting as an interaction medium for diffuse proton cosmic rays.
Nearby supernova remnants also produce gamma-rays as their expanding shell interacts with ambient dust.
The immediate area surrounding the galactic center a point-source emitter of gamma rays, though this mechanism is uncertain.

\subsection{Supermassive Black Hole}

Through kinematic observations of nearby stars, the galactic center is suspected to have a black hole, on the order of $10^6 M_{\odot}$. ??

\subsection{Diffuse Emission}
The disk of gas that permeates the galactic plane acts as an interaction medium for passing cosmic rays, both from galactic accelerators, as well as from extragalactic sources.
These high-energy protons collide with the dust, shattering into $\pi^{\pm,0}$.
The $\pi^0$ then decays into two gamma rays ??, providing the diffuse emission.
The spectrum of this emission is ??.


