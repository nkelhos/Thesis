\cleartooddpage[\thispagestyle{empty}]
\chapter{Gamma Rays}


\section{Methods of Creation}

There are three mechanisms relevant to this thesis that produce astrophysical gamma rays with TeV energies.
The leptonic mechanism is when gamma rays are accelerated via electrons, while the hadronic mechanism is when gamma rays are created via proton interactions.
The third mechanism, one explored in this thesis, is where two WIMP dark matter particles annihilate either directrly or indirectly into gamma rays.

In leptonic production, stars and galaxies have their own magnetic fields, and also emit electrons.
Electrons passing through these magnetic fields emit synchrotron photons, which can then scatter off of other electrons, gaining more energy. (reword this??)
After multiple cycles of this, a small fraction of the original photon population can gain TeV energies.
Other mechanisms can increase the number of TeV photons produced by a source, including stronger magnetic fields, breaking/reconnecting magnetic fields, as well as nearby ambient clouds of electrons.

In hadronic processes, protons are first accelerated by fermi acceleraton, emitted as part of a jet, or other mechanisms.
Then, upon striking an ambient atom, the proton will decay into $\pi^{+}$, $\pi^{-}$, and $\pi^{0}$.
The $\pi^{0}$ then quickly (<1sec) decays into a $\gamma\gamma$ pair, with \nicetilde $\frac{1}{10}$th the original proton's energy.
Much of the diffuse gamma ray component of the galactic disk is believed to be due to extra-galactic high-energy protons colliding with the atoms of the dusty galactic plane.(cite??)


\subsection{Dark Matter Interactions}
Indirect Detection: dark-dark => sm-sm
Direct Detection  : dark-sm => dark-sm
Collider Detection: sm-sm => dark-dark

Indirect can produce gamma rays via quark/lepton annihilation (??)
dark-dark => t-t => gamma-gamma
dark-dark => b-b => gamma-gamma


\section{Atmospheric Shower}
(this entire section needs more citations!??)

When a gamma ray, proton, or other particle strikes an atom of Earth's atmosphere, it can set off a cascade of energetic particles called an air shower.
When the primary particle is a gamma ray, an electron, or a positron, it creates an electromagnetic shower.
When the primary is a proton or other baryon, it creates a hadronic shower.
During either cascade of particles, any charged particles travelling at $v_{particle} > c_{atmosphere}$ will create Cherenkov photons, UV- and Visible-spectrum photons that are then imaged and recorded by the VERITAS observatory.

(image of particle cascade diagram??)

Electromagnetic air showers are started by a high energy (\nicetilde TeV) electron or gamma-ray, and produce a cascade of electrons, positrons, and photons, where initially each successive generation of particles tends to have more particles and less energy per particle than the last.
The primary gamma ray will interact with an atmospheric atom, producing a $e^{-}e^{+}$ pair, each with roughly half the primary gamma ray's energy.
The $e^{-}$ and $e^{+}$ will emit some photons through bremstrahlung radiation, until they only have a few MeV of kinetic energy, after which other energy loss mechanisms (compton, etc) dominate.
The photons created during the shower go on to produce more $e^{-}e^{+}$ pairs, though as each newly created particle has less energy than its parent particle, eventually the photons in the shower don't have enough energy to produce additional $e^{-}e^{+}$ pairs, and the shower dies off.

As most (\nicetilde 99\%??) of detected air showers are due to protons and not gamma rays, understanding the differences between hadronic showers and electromagnetic showers becomes useful in removing unwanted proton air showers and preserving gamma-ray air showers within the reconstruction software, sometimes referred to as gamma-hadron separation.
Hadronic showers start with a primary \nicetilde TeV proton that interacts with an atmospheric atom.
This proton then converts into a $\pi^{+}\pi^{-}\pi^{0}$, each with roughly \nicetilde 33\% (it varies though??) of the initial proton's energy.
The $\pi^{+}$ and $\pi^{-}$ can travel far from the main axis of the primary particle, then produce $\mu\nu$ pairs.
The $\pi^{0}$ quickly decays into $\gamma\gamma$, which then each start their own electromagnetic shower.
The $\pi^{+}$ and $\pi^{-}$ can have a large transverse momentum (relative to the primary shower axis), and also have longer decay time (??), both of which contribute to creating sub-cascades of showers further away from the primary particle axis, which tends to cause hadronic showers (and their resulting Cherenkov images) to be wider than a purely electromagnetic shower of the same length. 

(image of proton vs gamma shower??)

(image of proton vs gamma shower cherenkov image??)


\section{Galactic Center}
The galactic center is a complex region of space, with many astrophysical sources of gamma rays.
A disk of dust lies along the galactic plane, acting as an interaction medium for diffuse proton cosmic rays.
Nearby supernova remnants also produce gamma-rays as their expanding shell interacts with ambient dust.
The immediate area surrounding the galactic center a point-source emitter of gamma rays, though this mechanism is uncertain.

% black hole
Through kinematic observations of nearby stars, the galactic center is suspected to have a black hole, on the order of $10^6 M_{\odot}$. ??
The Galactic Center also is a source of TeV gamma rays, though the mechanism that produces them is still under debate.
% http://adsabs.harvard.edu/cgi-bin/bib_query?arXiv:1511.01159
One possibility is that the supermassive black hole emits particles, which convert into TeV gamma rays, while the second possibility suggests a nearby Pulsar Wind Nebula may be providing the gamma-ray-parent particles.
While the Galactic Center emits gamma rays, this emission is point-like to ground-based gamma ray telescopes. (rewrite this!??)
This analysis instead focuses on the gamma ray flux outside this point-like inner angular region. (citation!??)


The disk of gas that permeates the galactic plane acts as an interaction medium for passing cosmic rays, both from galactic accelerators, as well as from extragalactic sources.
These high-energy protons collide with the dust, shattering into $\pi^{\pm,0}$.
The $\pi^0$ then decays into two gamma rays ??, providing the diffuse emission.
The spectrum of this emission is ??.


