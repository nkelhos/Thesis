\cleartooddpage[\thispagestyle{empty}]
\newcommand{\Like}{L}
\newcommand{\LogLike}{\mathcal{L}}
\newcommand{\LogLikeMax}{\LogLike_{\textrm{max}}}
\newcommand{\xtrue}{x}
\newcommand{\xdet }{x'}
\newcommand{\ltrue}{l}
\newcommand{\ldet }{l'}
\newcommand{\btrue}{b}
\newcommand{\bdet }{b'}
\newcommand{\Etrue}{E}
\newcommand{\Edet }{E'}
\newcommand{\ttrue}{t}
\newcommand{\tdet }{t'}
\newcommand{\Aeff }{A_\textrm{eff }}
\newcommand{\Edisp}{E_\textrm{disp}}
\newcommand{\Mnull}{M_\textrm{null}}
\newcommand{\Malt }{M_\textrm{alt}}

\chapter{A Likelihood Search for Dark Matter}\label{chapter:analysis}

\section{VERITAS Data}\label{veritasdata}
The analysis in this thesis relies on three sets of VERITAS data.
One set contains observations of the Crab Nebula, and a second of the Galactic Center.
A third set contains observations of a dark region \nicetilde\ang{5} away from the Galactic Center.
This dark region is referred to as  Sgr A* Off, and is located at (l,b)=(\ang{357.3396}, \ang{3.9984}).
Sgr A* Off is located a few degrees away to avoid the bright diffuse gamma-ray emission caused by the galactic plane.
The Galactic Center and Sgr A* Off observation regions are shown in Figure~\ref{fig:gcfieldsofview}.
To quantify the cosmic-ray background, observations are taken at \ang{0.5} or \ang{0.7} offsets from each observing target, in four different directions (wobbles) along right ascension/declination axes.

\begin{figure}[!ht]
  \centering
  \includegraphics[width=0.85\textwidth]{images/skypointings/plot.pdf}
  \caption[VERITAS Galactic Center Pointings]{
    Fields of view for Galactic Center observations.
    Each circle marks the detection area of one telescope pointing.
    Green circles are Galactic Center observations, while dark blue circles are the Sgr A* Off observations used to construct the camera-background templates.
    The light blue band represents the galactic plane.
  }
  \label{fig:gcfieldsofview}
\end{figure}

All three sets of data include observations from both the V5 and V6 epochs (see Section~\ref{sec:epochs}), where the number of hours are shown in Table~\ref{tab:observation_times}.
All used data was taken from April 2010 to June 2016.
The specific VERITAS data run numbers are listed in Appendix~\ref{app:runlists}.

\begin{table}[!ht]
  \centering
  \caption{Hours of observations taken at each source/epoch combination.}
  \label{tab:observation_times}
  \begin{tabular}{|l|l|l|l|}
    \hline
    \textbf{Epoch} & \textbf{Crab Nebula} & \textbf{Sgr A*} & \textbf{Sgr A* Off} \\ \hline
    V5             & 3.3                  & 46.3            & 13.0                \\ \hline
    V6             & 5.5                  & 62.7            & 4.7                 \\ \hline
    % times calculated with $VERIPY/thesis/plots/obs_times.py
  \end{tabular}
\end{table}


There are comparatively fewer Sgr A* Off observations because this source is only used for background estimation, and telescope time is in high demand.
There are also fewer Crab Nebula observations, as the majority of its data is taken at higher elevations, where the telescope has increased sensitivity to lower energies.

For all of these observations, quality cuts are applied.
This includes monitoring the telescope hardware and cloud ceiling in the field of view.
Two far-infrared pyrometers are used to measure the cloud ceiling height, by measuring the temperature of the sky.
With the pyrometer, clouds (\nicetilde\ang{-5}C) are measured to be significantly warmer than clear sky (\nicetilde\ang{-50}C).
Low-quality segments of data, where there are large (>20\%) and/or rapid (<=\SI{30}{s}) changes in the L3 trigger rate or cloud ceiling height, are removed from the analysis~\cite{bird_weather}.
% see Bird 2015 thesis pg 55

Because each VERITAS epoch has a different hardware configuration, they also each have their own separate set of effective areas, point spread functions, energy migration matrices, and camera background models.
In addition, specific IRFs were calculated for additional data dimensions, including the frequency of night sky background photons in the camera, the telescope elevation, the event energy, and each event's distance from the camera center.
After this data is collected, it is used in a likelihood analysis, detailed in the next section.

\section{Likelihood Ratio Test}\label{sec:likeratio}
A likelihood ratio test determines the likelihood of getting the observed data from two separate model groups.
% Formally a likelihood analysis is frequentist in nature, 
% which means it tests the likelihood of the data given a model
%It is performed by calculating, for a group of events, the likelihood of detecting that group of events for each of two model groups.
Once the likelhood is calculated, then the parameters of the model are varied until the maximum likelihood is found for each model.
These two maximum likelihoods can then be used to calculate the test statistic (TS), which determines which model is statistically favored, and to what degree it is favored.

\subsection{Likelihood Calculation}
At its heart, a likelihood is a product of probabilities.
With two events, the likelihood is $(\textrm{probability that event one happens})\times(\textrm{probability that event two happens})$.
In a counting experiment like VERITAS, the likelihood $\Like$ is determined via Poissonian statistics with

\begin{equation}\label{eqn:simple_like}
  \Like = \frac{e^{-m} m^n}{n!} \,.
\end{equation}
In this equation, $n$ is the number of observed events, and $m$ is the average number of events predicted by a model group.

For increased statistical power, the VERITAS data can be split into bins of energy, galactic $l$ and $b$, and time.
When combining multiple bins with index $j$, each bin's likelihood is multiplied together as in Equation~\ref{eqn:simple_like_2}.
Future likelihood calculations may also include shower core position on the ground, or distance to the shower, or other observables.

%Equation~\ref{eqn:simple_like_2} is a general formula for calculating a binned likelihood of Poissonian events.
The general formula for calculating a binned likelihood of Poissonian events is

\begin{equation}\label{eqn:simple_like_2}
  \Like = \prod_j \frac{ e^{-m_j} \; m_{j}^{n_j}}{n_{j}!} \,.
\end{equation}
As events are grouped by bins, some information is lost, which generally results in a less powerful ratio test.
The result in Equation~\ref{eqn:simple_like_2} can be expanded into an unbinned likelihood through the following derivation.
First, Equation~\ref{eqn:simple_like_2} can be rearranged into

\begin{equation}\label{eqn:simple_like_4}
  \begin{aligned}
    \Like &= \prod_j e^{-m_j} \prod_j \frac{m_j^{n_j}}{n_{j}!} \\
    \Like &= e^{- \sum_j m_j} \prod_j \frac{m_j^{n_j}}{n_{j}!} \,.
  \end{aligned}
\end{equation}
Then, the size of each bin can be shrunk until there are only 1 or 0 events in each bin.
For empty bins (where $n=0$), the product in Equation~\ref{eqn:simple_like_4} becomes

\begin{equation}\label{eqn:simple_like_4a}
  n=0 \rightarrow \frac{m_j^{n_j}}{n_j!} = \frac{m_j^{0}}{0!} = \frac{1}{1} = 1 \,.
\end{equation}
For bins with 1 event, the product in Equation~\ref{eqn:simple_like_4} becomes

\begin{equation}\label{eqn:simple_like_4b}
  n=1 \rightarrow \frac{m_j^{n_j}}{n_j!} = \frac{m_j^1}{1!} = \frac{m_j}{1} = m_j = m_i \,,
\end{equation}
where $i$ is the $i^{\textrm{th}}$ event, and $m_i$ is the number of predicted events at event $i$'s sky position, energy, and time.
In this derivation, $m_j$ converts to $m_i$, because all the $n=0$ bins are now 1 and can be ignored, and thus a loop over the $j$ bins becomes a loop over the $i$ events.
Then Equation~\ref{eqn:simple_like_4} becomes

\begin{equation}\label{eqn:simple_like_5}
  \Like = e^{- \sum_j m_j} \prod_i m_i \,.
\end{equation}
Here, $\prod_i m_i$ encodes the data events, while $\sum_j m_{j}$ encodes the model information.

When calculating $\Like$, certain computational problems can arise.
Calculating the product of many small probabilities can result in extremely small numbers, beyond the binary storage limit of common variable types.
Calculating derivatives of some of these numbers is also computationally expensive, so to solve these two problems, the a logarithm is applied to get the log-likelihood:

\begin{equation}\label{eqn:simple_like_6}
  \LogLike = \textrm{log} \left ( \Like \right )  \,.
\end{equation}
This is possible because both $\Like$ and $\LogLike$ are strictly increasing functions, so the maximum of both will be at the same position in the parameter space.
The log-likelihood for a group of bins is then

\begin{equation}\label{eqn:simple_like_7}
  \LogLike = \textrm{log} \left ( e^{- \sum_j m_j} \prod_i m_i \right ) = - \sum_j m_j + \sum_i \textrm{log} \left ( m_i \right ) \,.
\end{equation}

When the bin size is infinitely small, $m_i$ becomes $P_i$, the value of the probability density function (of all models combined) at the position of the event via

\begin{equation}\label{eqn:simple_like_8}
  \LogLike = \textrm{log} \left ( e^{- \sum_j m_j} \prod_i m_i \right ) = - \sum_j m_j + \sum_i \textrm{log} \left ( P_i \right ) \,.
\end{equation}
This unbinned equation shows how the log-likelihood is calculated in this analysis.
The $\sum_{j} m_{j}$ term can be interpreted as the total number of events predicted by all models.

\subsection{Models}\label{sec:model_irf_folding}

Models are used in a likelihood analysis to predict the number of events that a particular source deposits into some bin $j$.
In a likelihood analysis, each source of events gets its own model.
Each model is described by a function $M$, and different sources will have different $M$ functions.
The function $M$ has dimensional units of $\frac{N}{A \; T \; E \; \Omega}$.
This function can be integrated over the energy, sky, and time region covered by a bin $j$ to calculate the number of events predicted in that bin, as shown by 

\begin{equation}\label{eqn:model_int}
  m_{j} = \int_{t\,\textrm{bin}} \int_{E\,\textrm{bin}} \int_{b\,\textrm{bin}} \int_{l\,\textrm{bin}} M(l,b,E,t)\; dl \; db \; dE \; dt \,.
\end{equation}
Here, $l$ and $b$ are sky coordinates, $E$ is energy, and $t$ is time.
The basic models used in this analysis can be broken apart into their spatial, spectral, and temporal components, as in

\begin{equation}\label{eqn:modelparts}
  M(l,b,E,t) = M_s(l,b,E,t) \; \; M_e(l,b,E,t) \; \; M_t(l,b,E,t) \,.
\end{equation}
For this thesis, the sources being modeled are in equilibrium, so time-dependent effects are ignored by setting $M_{t}(l,b,E,t) = 1$.

For a basic point source, the spatial model function $M_s$ is

\begin{equation}\label{eqn:pntsrc_Ms}
  M_{s,\textrm{point}}(l,b) = \lim_{a\to\infty} \frac{1}{ \abs{a} \sqrt{\pi} } e^{ - \left ( \sqrt{ (l-l_o)^2 + (b-b_o)^2 }/a \right )^2 } \,.
\end{equation}
In this equation, $l_o$ and $b_o$ specify the position of the point source in galactic sky coordinates, which can be model parameters.
More complex $M_s$ functions may also depend on energy $E$ or time $t$, or may take on other shapes like an ellipse, a ring, or any other two-dimensional function.
The energy spectrum can be similarly modeled by a basic power law.
This basic power law is defined by the model function

\begin{equation}\label{eqn:powerlaw_Me}
  M_{e,\textrm{powerlaw}}(E) = N_o \left ( \frac{E}{E_o} \right )^{-\gamma} \,.
\end{equation}
Here, $N_o$ is the flux normalization, $E_o$ is the pivot energy, and $\gamma$ is the spectral index, which are all be model parameters.


\subsection{Instrument Response Function Folding}\label{subsec:folding}
The average number of counts predicted by a model is calculated by integrating over the entire energy, sky, and time regions in the analysis.
Because the reconstruction method is not perfect, all events from the astrophysical models are diffused according to the PSF and energy dispersion.
Another way of understanding this is if a source is located in some sky bin $j$, events from that source can be reconstructed in neighboring bins $j+1$ and $j-1$.
This increases the predicted number of events in those bins and decreases the number of events in bin $j$.
The PSF is then a function that describes how many and how far events are diffused away from a sky position.
This dispersion is illustrated in Figure~\ref{fig:responsedispersion}.
Similarly, energy dispersion will diffuse events out among neighboring energy bins.
The effective area also alters the detector response, leading to more collection area at higher energies, due to being able to detect the brighter showers of high-energy gamma rays from further away.
The PSF, energy dispersion, and effective area are collectively known as instrument response functions.

\begin{figure}[!t]
  \centering
  \includegraphics[width=0.85\textwidth]{images/responsefunction/responsefunction.pdf}
  \caption[Response Function Dispersion]
  {
    Left: Events from a source at the green arrow detected by a `perfect' detector, with bins in true coordinates.
    Right: The same events detected by a `real-world' detector, with bins in reconstructed coordinates.
  }
  \label{fig:responsedispersion}
\end{figure}

This leads to the need to define two distinct coordinate systems, $\xtrue$ and $\xdet$.
The coordinate $\xtrue$ is in true space (galactic $\ltrue$ and $\btrue$, energy $\Etrue$, and time $\ttrue$), \textit{before} the IRFs are applied.
Alternatively, $\xdet$ is the coordinate in reconstructed detector space ($\ldet$ and $\bdet$, $\Edet$, and $\tdet$), \textit{after} the IRFs are applied.
Another way this can be thought of is that $\xtrue$ is the physical coordinate for events \textit{before} they reach Earth's atmosphere, and $\xdet$ is \textit{after} the events have been reconstructed.
Applying the IRFs is called \textit{folding}, and is performed by an integration:

\begin{equation}\label{eqn:folding}
  P_i \left( \xdet \right ) = \int_\xtrue R \left ( \xdet, \xtrue \right ) * M \left ( \xtrue \right ) d\xtrue \,.
\end{equation}
Here, $P_i \left( \xdet \right )$ is the probability of detecting an event at detector coordinates $\xdet = \left ( \ldet, \bdet, \Edet, \tdet \right )$.
The integration $\int_\xtrue$ takes place over the entire true space, time, and energy regions being studied.
The function $M\left ( \xtrue \right )$ is the number of counts predicted by the astrophysical models at coordinate $\xtrue$.
The function

\begin{equation}\label{eqn:foldingR}
  R(\xdet,\xtrue) = \Aeff(\xtrue) * PSF(\xdet,\xtrue) * \Edisp(\Edet,\xtrue) \,,
\end{equation}
is the instrument response function, which incorporates the effective area, PSF, and energy dispersion information, and is discussed further in Sections \ref{subsec:effarea}, \ref{subsec:psf}, and \ref{subsec:edisp}.
The functions $\Aeff$, $PSF$, and $\Edisp$ are all interpolated from tables of stored values, which are derived from simulations.

The Crab Nebula point source in Section~\ref{sec:crab_analysis}, the Galactic Center point source in Section~\ref{subsec:gcpointsrc}, and the dark matter halo model in Section~\ref{subsec:dmhalomodel} all have this folding applied to the number of events they predict.
An important distinction is that this folding is only applied to these astrophysical models, and not to the camera background models.
This is because the background models are created with data from actual observations, and thus are already in $\xdet$ space.

\subsection{Combining Models into Hypotheses}\label{subsec:hypotheses}

When calculating the likelihood of a bin in Equation~\ref{eqn:simple_like}, the predicted number of counts in a bin may come from a combination of sources.
Some fraction may come from a background model, another fraction from a specific source model, and some from other models.
So in order to account for these multiple models, their predicted counts in a bin must be summed first, as in 

\begin{equation}\label{eqn:combinemodels}
  m_{j} = \sum_k m_{k,j} \,,
\end{equation}
before being used in Equation~\ref{eqn:simple_like_8}.
Here, index $k$ loops over the various models that contribute events at a position in the parameter space.
The factor $m_{k,j}$ represents the number of counts predicted by model $k$, at the position of bin $j$.

In order to calculate the test statistic, the models must be grouped into two sets, called hypotheses.
For a basic analysis, the null hypothesis consists of all models, except the specific one being searched for, called here $H_{\textrm{null}}$.
The second hypothesis, called the alternate hypothesis, consists of all the models in the null hypothesis, plus the model being searched for, called here $H_{\textrm{alt}}$.
Each observation would get its own camera background model, and then any additional astrophysical models are added separately.
For example, if one has three observations of the Crab Nebula, this would mean there are three camera background models, plus a point source model for the Crab Nebula.
The null hypothesis would be just the camera background models, while the alternate hypothesis would be the camera backgrounds plus the Crab Nebula model.
Once these two hypotheses are assembled for an analysis, their maximum likelihood can then be sought.

\subsection{Likelihood Maximization}\label{subsec:likemax}
The maximum likelihood is found by iteratively changing the parameters of a hypothesis's component models in directions that increase the likelihood.
For example, take the alternate hypothesis used in Section~\ref{subsec:hypotheses}, with three camera background models and one Crab Nebula point source.
Each camera background model has a base template multiplied by a power law.
The camera background models naively only need to fit the normalization, but the effect of using templates from one elevation with data at another elevation was uncertain, so the spectral index was also allowed to change.
Each power law has two parameters, a normalization and a spectral index, so the background camera models have the parameters $N_1$, $N_2$, $N_3$, $\gamma_1$, $\gamma_2$, and $\gamma_3$.
The Crab Nebula point source power law also has normalization and spectral index parameters $N_c$ and $\gamma_c$.
The Crab Nebula model also has location parameters $l_c$ and $b_c$, but since these are fixed in this example (and thesis), the likelihood maximization can't vary them.
Therefore this alternate hypothesis would have 8 free parameters, $N_1$, $N_2$, $N_3$, $N_c$, $\gamma_1$, $\gamma_2$, $\gamma_3$, and $\gamma_c$.
For calculating the test statistic for the presence of the Crab Nebula, the null hypothesis is just the camera background models, with 6 free parameters $N_1$, $N_2$, $N_3$, $\gamma_1$, $\gamma_2$, and $\gamma_3$.

When finding the maximum likelihood for each hypothesis, these free parameters are varied, and the likelihood is recalculated.
This procedure is repeated until a maximum likelihood is reached.
While there are many maximization algorithms, this analysis uses the Levenberg-Marquardt method~\cite{marquardt1963algorithm}.
Once the maximum likelihood is calculated for both hypotheses, then the test statistic can be calculated.

\subsection{Test Statistic Calculation}

In order to search for the presence of a source, the test statistic (TS) determines how favorable the alternate hypothesis is compared to the null hypothesis.
Once the maximum likelihood $\LogLike_{\textrm{max}}$ is found for these two hypotheses, the TS can be calculated with

\begin{equation}\label{eqn:tscalc}
  \textrm{TS} = - 2 \; \textrm{log} \left (  \frac{ \LogLikeMax( \Mnull ) }{ \LogLikeMax( \Malt ) } \right ) \,.
\end{equation}

To convert this TS into a p-value, events can be repeatedly simulated using the null hypothesis probability density function.
Then for each simulation, the TS is calculated.
After many simulations, Wilk's theorem~\cite{wilks1938} says the resulting TS's will form a $\chi^2$ distribution with $n$ degrees of freedom, where $n$ is the difference in number of free parameters between the two hypotheses.
From this simulated TS distribution, an actual TS can be converted into a p-value.
In situations where there is only one or two degrees of freedom, the significance of a specific model can be calculated as $\sqrt{\textrm{TS}}$.
Due to time constraints, the p-value of this test statistic was not calculated.
% http://pulsar.sternwarte.uni-erlangen.de/black-hole/2ndschool/talks/likelihood_1.pdf


\section{Background Models}\label{sec:bkgmodels}
The background models predict the number of background counts produced by a sky without gamma rays.
This is used to model the probability density function of the background (primarily proton) events, which are several orders of magnitude more populous than the gamma rays.
Background models are produced by binning events from observations of sky positions that have with weak or no gamma-ray emission.
This results in a template of how the gamma-like background proton events are distributed in the camera.
The construction of these models is described further in Section~\ref{background_production}.
For this low-elevation analysis the observations of the dark region Sgr A* Off, described in Section~\ref{veritasdata}, were used to build these backgrounds.

To account for the difference between the V5 and V6 observatory configurations, the background observations are divided up based on their VERITAS hardware epoch, producing a unique background template for each epoch.
These background templates depend on the event's position in the camera and the event energy.
They are used in both the Crab Nebula analysis and the Galactic Center analysis.
Because the templates are made from Sgr A* Off, which has no sources of gamma-rays, the only expected detectable events are due to background cosmic rays.

\section{Crab Nebula Likelihood Analysis}\label{sec:crab_analysis}
To verify the likelihood method is physically correct, the Crab Nebula was analyzed first, before any dark matter analysis was performed.
As the Crab Nebula is the brightest gamma ray emitter in the sky, it has been observed extensively by VERITAS and other gamma ray telescopes.
After searching for low-elevation Crab Nebula observations, a total of 17.1 hours of data were selected from the VERITAS data archives.

Since the Galactic Center only rises to around \ang{29} elevation, low-elevation effects should be investigated.
A plot of the telescope pointing elevation for the Crab Nebula, Galactic Center, and Galactic Center off data is shown in Figure~\ref{fig:datapointingelevations}.
To uncover any low-elevation effects, time cuts were applied to this data to restrict the telescope pointing elevations to \SIrange{27.5}{32.5}{\degree}, similar to the elevation of the later Galactic Center data.
This resulted in \SI{3.3}{hours} of V5 and \SI{5.5}{hours} of V6 epoch data (see Table \ref{tab:observation_times}).
In Figure~\ref{fig:crab_skymap}, the position of all counts is shown in galactic $\ldet$ and $\bdet$.

\begin{figure}[hb]
  \centering
  \includegraphics[width=0.75\textwidth]{images/elevation_hist/elevhist.pdf}
  \caption[VERITAS Data Elevation Exposure]{
    Camera center elevation for the three sets of data.
    The three peaks in the Sgr A* data are from the 4 wobble positions being at different elevations.
    The north wobble observations peak at elevation \nicetilde\ang{29.75}, east and west wobbles observations at \nicetilde\ang{29.25}, and south wobble observations at \nicetilde\ang{28.75}.
  }
  \label{fig:datapointingelevations}
\end{figure}

  
\begin{figure}[bt]
  \centering
  %\includegraphics[width=0.85\textwidth]{images/test_crab_analysis/plot_elev27_5_32_5deg_4_70TeV_wobbleall_Epochall_skymap.pdf}
  \includegraphics[width=0.65\textwidth]{images/test_crab_analysis/plot_elev27_5_32_5deg_4_70TeV_nfits165_mapcounts.pdf}
  \caption[Crab Nebula Counts Sky Map]
  {
    Sky map of event positions, ($\ldet$,$\bdet$).
    No corrections are made for observing time or effective area.
  }
  \label{fig:crab_skymap}
\end{figure}

The Crab Nebula is modeled by a point source with the simple power law spectrum,

\begin{equation}\label{eqn:crab_model}
  M(\xtrue) = M_{s,\textrm{powerlaw}}(\xtrue) * M_{e,\textrm{point}}(\xtrue) = N_o \left ( \frac{\Etrue}{E_o} \right )^{-\gamma} * \lim_{a\to\infty} \frac{1}{ \abs{a} \sqrt{\pi} } e^{ - \left ( \sqrt{ (\ltrue-l_c)^2 + (\btrue-b_c)^2 }/a \right )^2 } \,.
\end{equation}
These spatial and spectral shapes are both discussed in Section~\ref{subsec:likemax}.
The sky position of the point source is fixed to the Crab Nebula, at

$$(l_c,b_c)_{\textrm{J2000}} = (\ang{184.557600},\ang{-5.784180}) \;\;.$$
The pivot energy $E_o$ of its spectrum is fixed at \SI{16.73}{\TeV{}}, while the normalization $N_o$ and the spectral index $\gamma$ are free to vary during the likelihood optimization.

Only events between \SIrange{4}{70}{\TeV{}} are used in this test analysis.
At an elevation of \ang{25}, the reconstruction method is able to reconstruct events as low as \SI{1.5}{\TeV{}}.
Below \SI{4}{\TeV{}}, however, the camera sensitivity starts to decrease in a poorly understood way, and IRFs in this region may not be accurate.
Part of this decrease is explored in Section~\ref{subsec:bkgstructure} (see Figures \ref{fig:bkgvsel_crab} and \ref{fig:bkgvsel_sgra}).
The analysis range was limited to \SI{70}{\TeV{}} due to only having accurate IRFs up to that energy.
  
% values from nkelhos@warp-zeuthen.desy.de:/afs/ifh.de/group/cta/scratch/nkelhos/dm_halo_testing/veripy/thesis/analysis/crab_test/logs/statistics.txt
After fitting all model parameters to the events from \SIrange{4}{70}{\TeV{}}, the best fit power law values are $ N_o = \left(3.90\pm0.71\right)*10^{-20} \frac{\textrm{photons}}{\textrm{cm}^{2} \; \textrm{s} \; \textrm{MeV} } $, $ \gamma = 2.31 \pm 0.17 $, with a test statistic of 408.8, corresponding to \nicetilde{}\SI{20.2}{$\sigma$}.
%As the alternate Crab Nebula hypothesis has 110 free parameters, and the no-Crab-Nebula (null) hypothesis has 108 free parameters, the test statistic has $ 110 - 108 = 2 $ degrees of freedom.
% $ python
% >>> from scipy import stats
% >>> 1 - stats.chi2.cdf(4,1)
% 0.04550026
% >>> 1 - stats.chi2.cdf(40,1)
% 2.539628507491898e-10
% >>> 1 - stats.chi2.cdf(400,1)
% 0.0
% (super small number)


\begin{figure}[!t]
  \centering
  \includegraphics[width=0.95\textwidth]{images/test_crab_analysis/plot_elev27_5_32_5deg_4_70TeV_nfits165_spectra.pdf}
  \caption[Crab Nebula Spectra]
  {
    Crab Nebula spectra from various analyses and observatories.
    The solid red line is the best-fit spectra from the CTOOLS analysis described in this chapter, using only events from \SIrange{4}{70}{\TeV{}}.
    The inner red envelope is the statistical fitting error on the solid red line.
    The outer red envelope is the combined statistical+systematic uncertainty.
    The dark blue line is the standard VERITAS Eventdisplay spectrum using the same set of observations.
    The dark blue data points are flux points for specific energy bins, from Eventdisplay.
    Light blue is a Crab Nebula spectrum from HESS~\cite{hess2006crab}.
    Purple is a previously published spectrum from VERITAS~\cite{veritas2015crab}.
    Orange is a spectrum from MAGIC~\cite{magic2015crab}.
  }
  \label{fig:crab_test_spectra}
\end{figure}
  
\begin{table}[!t]
  \centering
  \begin{tabular}{|r|c|c|c|c|c|}
    \hline
    \textbf{Analysis} & \textbf{Min}    & \textbf{Max}    & \textbf{FOV} & \textbf{PSF} & \textbf{$\sigma$} \\
    \textbf{Method}   & \textbf{Energy} & \textbf{Energy} &  \#          & \#           &                   \\
                      & TeV             & TeV             &  Events      & Events       &                   \\
    \hline 
    On/Off Region & 3.16 & 79.4 & 11197 & 145 & 21.3 \\
    Likelihood    & 4.00 & 70.0 & 9319  & 120 & 20.2 \\
    \hline 
  \end{tabular}
  \caption[Analysis Comparison]{
    Comparison between the two different Crab Nebula analyses, using the ON/OFF regions in Eventdisplay, and the Likelihood analysis in ctools.
  }
  \label{tab:crab_statistics}
\end{table}

In the standard VERITAS Eventdisplay analysis, the Crab Nebula is found to have a point source significance of \SI{21.3}{$\sigma$}, shown in Table~\ref{tab:crab_statistics}.
However, the energy range of this Eventdisplay analysis was from \SIrange{3.16}{79.4}{\TeV{}}, which contained a total of 11197 events in the field of view.
The likelihood analysis was from \SIrange{4}{70}{\TeV{}}, containing only 9319 events, \nicetilde17\% fewer events.
This ratio persisted when the events were limited to within \ang{0.18} of the Crab Nebula, the approximate radius of the PSF at \SI{10}{\TeV{}} (145 vs 120 events).
Thus, with \nicetilde17\% fewer events, the likelihood test detects the Crab Nebula at a similar significance level, implying this likelihood method is more sensitive than the standard analysis alone.
% event display: 3.16 TeV to 79.4 TeV
% ctools       : 4    TeV to 70   TeV
% from nkelhos@warp-zeuthen.desy.de:~/dm_halo_testing/veripy/thesis/analysis/crab_test/energy_range_event_count_difference.py
In Figure~\ref{fig:crab_test_spectra}, the fitted Crab Nebula spectra is shown, along with literature results from earlier VERITAS, HESS, and MAGIC observations of the Crab Nebula.
The fitted spectra from this work is shown as a red line, with a red statistical uncertainty band.
A second larger red uncertainty band includes a \nicetilde{}20\% systematic uncertainty on the flux, due to gamma rays of different energies producing similar-looking shower images in the VERITAS telescopes.

\begin{figure}[p]
  \centering
  \includegraphics[width=0.85\textwidth]{images/test_crab_analysis/plot_elev27_5_32_5deg_4_70TeV_nfits165_profiledata_spatiall_json.pdf}
  \llap{
    \makebox[13cm][l]{ % x position
      \raisebox{5.2cm}{     % y position
        {
          \setlength{\fboxsep}{0pt}
          \setlength{\fboxrule}{1pt}
          \fbox{
            \includegraphics[height=4cm]{images/test_crab_analysis/plot_elev27_5_32_5deg_4_70TeV_nfits165_profl_skymap_rasterized.pdf}
          }
        }
      }
    }
  }
  \caption[Crab Nebula Profile along Galactic $l$]
  {
    The top plot shows the number of counts along a \ang{0.15}-wide-slice through Crab Nebula, along the galactic $l$ axis.
    Blue points are the number of observed counts, with Poissonian error bars~\cite{poissonfrequentistinterval}.
    The green histogram bars are the number of counts predicted by all models.
    Yellow histogram bars are the number of counts predicted by only the camera-background models.
    The bottom plot shows the $\frac{\mathrm{all\:models}}{\mathrm{data}}$ residual as green points.
    The inset plot shows the counts map from Figure~\ref{fig:crab_skymap}, with blue squares showing the profile bin locations.
  }
  \label{fig:crab_profile_l}
\end{figure}

\begin{figure}[p]
  \centering
  \includegraphics[width=0.85\textwidth]{images/test_crab_analysis/plot_elev27_5_32_5deg_4_70TeV_nfits165_profiledata_spatialb_json.pdf}
  \llap{
    \makebox[13cm][l]{ % x position
      \raisebox{5.2cm}{  % y position
        {
          \setlength{\fboxsep}{0pt}
          \setlength{\fboxrule}{1pt}
          \fbox{
            \includegraphics[height=4cm]{images/test_crab_analysis/plot_elev27_5_32_5deg_4_70TeV_nfits165_profb_skymap_rasterized.pdf}
          }
        }
      }
    }
  }
  \caption[Crab Nebula Profile along Galactic $b$]
  {
    The top plot shows the number of counts along a \ang{0.15}-wide-slice through the Crab Nebula along the galactic $b$ axis.
    Blue points are the number of observed counts, with Poissonian error bars~\cite{poissonfrequentistinterval}.
    The green histogram bars are the number of counts predicted by all models.
    Yellow histogram bars are the number of counts predicted by only the camera-background models.
    The bottom plot shows the $\frac{\mathrm{all\:models}}{\mathrm{data}}$ residual as green points.
    The inset plot shows the counts map from Figure~\ref{fig:crab_skymap}, with blue squares showing the profile bin locations.
  }
  \label{fig:crab_profile_b}
\end{figure}
    
\begin{figure}[p]
  \centering
  \includegraphics[width=0.85\textwidth]{images/test_crab_analysis/plot_elev27_5_32_5deg_4_70TeV_nfits165_profiledata_spatiallalt_json.pdf}
  \llap{
    \makebox[13.2cm][l]{ % x position
      \raisebox{6.8cm}{  % y position
        {
          \setlength{\fboxsep}{0pt}
          \setlength{\fboxrule}{1pt}
          \fbox{
            \includegraphics[height=2.4cm]{images/test_crab_analysis/plot_elev27_5_32_5deg_4_70TeV_nfits165_profa_skymap_rasterized.pdf}
          }
        }
      }
    }
  }
  \caption[Crab Nebula Profile along Galactic $l$ Off Source]
  {
    The top plot shows the number of counts along a \ang{0.15}-wide-slice along the galactic $l$ axis.
    This slice doesn't go through the Crab Nebula, but instead \ang{1} higher in galactic $b$.
    As this doesn't include the Crab Nebula, this plot primarily demonstrates the camera background modeling.
    Blue points are the number of observed counts, with Poissonian error bars~\cite{poissonfrequentistinterval}.
    The green histogram bars are the number of counts predicted by all models.
    As the Crab Nebula doesn't contribute any events \ang{1} to the north, the green histogram bars are identical (and behind) the yellow histogram bars.
    Yellow histogram bars are the number of counts predicted by only the camera-background models.
    The bottom plot shows the $\frac{\mathrm{all\:models}}{\mathrm{data}}$ residual as green points.
    The inset plot shows the counts map from Figure~\ref{fig:crab_skymap}, with blue squares showing the profile bin locations.
  }
  \label{fig:crab_profile_l_off}
\end{figure}

\begin{figure}[p]
  \centering
  \includegraphics[width=0.85\textwidth]{images/test_crab_analysis/plot_elev27_5_32_5deg_4_70TeV_nfits165_profiledata_energy_json.pdf}
  \llap{
    \makebox[9.5cm][l]{  % x position
      \raisebox{5.8cm}{ % y position
        {
          \setlength{\fboxsep}{0pt}
          \setlength{\fboxrule}{1pt}
          \fbox{
            \includegraphics[height=3.5cm]{images/test_crab_analysis/plot_elev27_5_32_5deg_4_70TeV_nfits165_profe_skymap_rasterized.pdf}
          }
        }
      }
    }
  }
  \caption[Crab Nebula Profile in Energy]
  {
    The top plot shows the number of counts in a \ang{0.6} x \ang{0.6} square centered on the Crab Nebula, vs energy.
    Blue points are the number of observed counts, with Poissonian error bars~\cite{poissonfrequentistinterval}.
    The green histogram bars are the number of counts predicted by all models.
    Yellow histogram bars are the number of counts predicted by only the camera-background models.
    The bottom plot shows the $\frac{\mathrm{all\:models}}{\mathrm{data}}$ residual as green points.
    The inset plot shows the counts map from Figure~\ref{fig:crab_skymap}, with a blue square showing the profile bin location.
  }
  \label{fig:crab_profile_energy}
\end{figure}
    
The fitted models can also be viewed as a check that the likelihood engine is fitting the models to the data.
In Figures \ref{fig:crab_profile_l} and \ref{fig:crab_profile_b}, the counts from the observations and models were integrated along a slice of galactic $l$ and $b$.
The counts from only the camera background models is shown in yellow, along with the counts from all camera backgrounds plus the point source in green.
The difference between these two histograms is then the counts from the Crab Nebula point source model.
In Figure~\ref{fig:crab_profile_l_off}, a profile is made at \ang{1} north of the Crab Nebula in Galactic $b$.
This shows how when no source is present, the background is still well modeled by the background models.
In Figure~\ref{fig:crab_profile_energy}, a similar plot is made, though integrated in a \ang{0.6}$\times$\ang{0.6} square around the Crab Nebula at different energies.
  
\begin{figure}[tb]
  \centering
  \includegraphics[width=0.50\textwidth]{images/test_crab_analysis/plot_elev27_5_32_5deg_4_70TeV_nfits165_mapresiduals_coarse.pdf}
  \caption[Crab Residual Sky Map]{
    Sky map showing how significantly the models differ from the observations in each bin.
    This plot uses \ang{0.23}-wide bins.}
  \label{fig:crab_signif_skymap_coarse}
\end{figure}
  

For the Crab Nebula, a sky map of the significances are shown in Figures~\ref{fig:crab_signif_skymap_coarse}.
The significance of each sky bin is calculated with the equation

\begin{equation}\label{eqn:resmap_signif}
  \textrm{Significance} = \textrm{sign}(D-M) \times \sqrt{ 2 \left ( D \: \textrm{ln} \left ( \frac{D}{M} \right ) + M - D \right ) } \,,
\end{equation}
which is derived in Appendix~\ref{app:sigdist}.

When a significance distribution is made of Figure~\ref{fig:crab_signif_skymap_coarse}'s pixels, it is seen in Figure~\ref{fig:crab_signif_distribution} to follow a Gaussian distribution.
Simulations are shown in red using the best-fit models, and the $\frac{\textrm{Simulated Models}}{\textrm{Data}}$ residual is shown in the bottom plot.
Since most bins in the residual overlap 1, it can be concluded that the models are not deficient in any particular area.

\begin{figure}[tb]
  \centering
  \includegraphics[width=0.55\textwidth]{images/test_crab_analysis/plot_elev27_5_32_5deg_4_70TeV_nfits165_mapresiduals_coarse_hist.pdf}
  \caption[Crab Residual Bin Distribution]{
    Data and simulated significance distributions from the fine-binned significance sky map in Figure~\ref{fig:crab_signif_skymap_coarse}.
    In the top plot, the gray histogram bars indicate the data's significance distribution, with gray Poissonian error bars.
    The red lines and bars indicate the range of simulated significances, centered on the mean, and extending up and down by one standard deviation.
    The bottom plot shows the $\frac{\textrm{simulated}}{\textrm{data}}$ residual.
  }
  \label{fig:crab_signif_distribution}
\end{figure}
  
However, this distribution is not quite Gaussian-shaped, due to how the significance is calculated.
Because this is tangential to the analysis, it is only discussed further in Appendix~\ref{app:sigdist}.

\FloatBarrier

\section{Dark Matter Likelihood Analysis}\label{sec:dmlike}
  
Since the test analysis on the Crab Nebula data shows results consistent with other VERITAS, H.E.S.S., and MAGIC studies, the main dark matter analysis can begin in earnest.
The 108 hours of Galactic Center data used in this analysis is described in Section~\ref{veritasdata}.
A sky-map histogram of all observed events is shown in Figure~\ref{fig:gc_counts_skymap}.
This sky map is uncorrected for exposure time or effective area; it is only a histogram of event positions.
A histogram of all events' energies from \SIrange{4}{70}{\TeV{}} is shown in Figure~\ref{fig:gc_counts_enhist}.
As this histogram is uncorrected for effective area, it does not follow the standard power-law shape.
Once the data is reconstructed, the next step is to set up the models.
These include the camera background models, similar to the Crab Nebula analysis, as well as a point source at the Galactic Center, and a dark matter halo.
  
\begin{figure}[bt]
  \centering
  \includegraphics[width=0.65\textwidth]{images/likelihood_analysis/plot_brbbbar_45_00TeV_counts_skymap.pdf}
  \caption[Galactic Center Counts Sky Map]{
    Sky map of all events used in this analysis.
    No adjustments are made here for effective area, observation time, or background rate.
  }
  \label{fig:gc_counts_skymap}
\end{figure}

\begin{figure}[tb]
  \centering
  \includegraphics[width=0.65\textwidth]{images/likelihood_analysis/plot_brbbbar_45_00TeV_enhist.pdf}
  \caption[Galactic Center Counts Energy Histogram]{
    Histogram of all event energies ($\Edet$) used in this analysis.
    No adjustments are made here for effective area, observation time, or background rate.
  }
  \label{fig:gc_counts_enhist}
\end{figure}

\FloatBarrier

\subsection{Non-Dark Astrophysical Models}\label{subsec:gcpointsrc}
For this analysis, a point source model was added at the position of Sgr A*, at

$$(l,b)_{\textrm{J2000}} = (\ang{359.944212}, \ang{-0.046013}) \,. $$
This was done because previous studies had detected a gamma-ray excess at Sgr A*, and it was found to be point-like.
This excess may be produced by WIMPs annihilating, or it may be due to some other non-dark-matter mechanism (see Section~\ref{sec:gc}).
In this analysis, the excess from Sgr A* is considered as not coming from dark matter and a point source is added.
%If the excess is due to a non-dark-matter mechanism, the excess emission must be accounted for by a point source model in the likelihood analysis, so that the excess events do not influence the the dark matter halo model's search.
%However, if the excess is due to WIMP annihilation, a point source model should not be added.
%This leads to a choice of model configurations, whether to add a point source model or not.

Adding a point source produces a more conservative upper limit on the velocity-averaged cross section $\vacs$.
This occurs because, assuming the non-dark-matter scenario but without a point source model, there are more unmodeled events.
When increasing the $\vacs$, more unmodeled events cause the likelihood to decrease at an earlier (smaller) $\vacs$, which results in a smaller upper limit (see Section~\ref{upper_limit}) that eliminates more of the parameter space.
Therefore the $\vacs$'s excluded by the conservative (with-point-source) upper limit are true for both WIMP and pulsar scenarios.
{\color{red}(this would not really lead to a more conservative upper limit though, right? if you would have considered the excess as coming from dark matter, you would have claimed a discovery, not set an upper limit, no?? -orel)}

% Hooper at TeVPa 2018: 
%   Event:  https://indico.desy.de/indico/event/18204/session/12/contribution/132
%   Slides: https://indico.desy.de/indico/event/18204/session/12/contribution/132/material/slides/0.pdf
% Idea: if no point source, more 'unexplained' photons that dont fit the halo push the upper limit lower,
%   so if a point source is added to explain some of the events, there will be fewer 'unexplained' events
The point source's spectral shape is set by a broken power law,

\begin{equation}\label{eqn:brokenplaw}
  M_e(\xtrue) = M_e(\Etrue) = N_o * { \left ( \frac{\Etrue}{E_{\textrm{pivot}}} \right ) }^{\gamma} {e}^{-\frac{E}{E_{\textrm{cutoff}}}} \,\,.
\end{equation}
This was chosen because, in a previous VERITAS analysis of the Galactic Center, a broken power law was found to be a better fit than a simple power law~\cite{VeritasGCRidge2015}.
  
Several parameters are fixed in the likelihood optimization.
The values used in Equation~\ref{eqn:brokenplaw} are from Ref. \cite{VeritasGCRidge2015}, where $E_{\textrm{pivot}}=\SI{1}{TeV}$, $E_{\textrm{cutoff}}=\SI{12.8}{TeV}$, and $\gamma=-2.1$.
% theres also the 2016 paper http://iopscience.iop.org/article/10.3847/0004-637X/821/2/129/meta
The normalization parameter $N_o$ was initially set to $2.8*{10}^{-12}\,\text{cm}^{-2}\,\text{s}^{-1}\,\text{TeV}^{-1}$, but was free to change in the likelihood optimization, while $E_{\textrm{pivot}}$, $E_{\textrm{cutoff}}$, and $\gamma$ were all fixed.
The normalization $N_o$ was left free to allow for the potential of some mixing between the point source events and any dark matter halo events.
For example, a stronger dark matter halo gamma-ray flux would result in a weaker point source flux.

The galactic disk also produces its own gamma-ray emission, as the protons in the disk act as an interaction target for relativistic protons~\cite{tevgev_gc_diffuse}.
At these high energies, the contribution from the galactic diffuse emission is negligible compare to atmospheric effects discussed in Section~\ref{sec:elevgradient}.

This gradient is further discussed in Section~\ref{sec:elevgradient}.
  
\subsection{Dark Matter Models}\label{subsec:dmhalomodel}
Dark matter halos are modeled by a spherically-symmetric mass-per-volume density profile, combined with an annihilation spectrum.
This is assembled with a simplified version of Equation \ref{eqn:dmflux}:

\begin{equation}\label{eqn:dmmodel}
  M_{\textrm{dm}} = N \times M_{\textrm{e,halo}} \times M_{\textrm{s,halo}} \,.
\end{equation}
In this analysis, an Einasto density profile is used for the spatial component function $M_s$ of the dark matter halo model.
See Section~\ref{dm_spatial} for a discussion on the $M_{s,\textrm{halo}}$ function.

For the spectral component, each of the nine dark matter masses tested has their own spectrum function $M_e$ produced with CLUMPY.
These spectra are shown in Figure~\ref{fig:chichi_spectrum2}, and are discussed further in Section~\ref{dm_spectral}.
The parameter $N$ is the magnitude of the halo, which was left free in the likelihood fit.

\begin{figure}[bt]
  \centering
  \includegraphics[width=0.55\textwidth]{images/spectra/chichi_spectrum.pdf}
  \caption[Single Annihilation Spectra]{
    Resultant photon spectra from the annihilation of two WIMP particles purely into the $b\bar{b}$ channel.
    Each colored line represents a different WIMP mass.
  }
  \label{fig:chichi_spectrum2}
\end{figure}

\FloatBarrier

\subsection{Likelihood Maximization Results}\label{like_results}

The fully assembled likelihood analysis is described in the following section.
108 hours of VERITAS data have been organized into 948 observations.
Each observation has a camera background model, with two free parameters, a normalization $N_o$ and a spectral index $\gamma_o$.
In this analysis as well, the backgrounds are a template multiplied by a power law, in case the spectrum changes due to the changing elevation of the observations.
A point source has been added at the Galactic Center with a broken power law spectrum, where only its normalization $N_o$ is a free parameter.
The last model is then one of nine dark matter halo models, each with a cuspy Einasto spatial profile.
The spectrum of this halo is from two WIMPs of mass $m_{\chi}$ annihilating into a $b\bar{b}$ pair.
The dark matter halo's only free parameter is its normalization $N_o$.
These models are then grouped into two hypotheses, where $\Mnull$ consists of the 948 camera background models and the Galactic Center point source.
Then $\Malt$ is all the models in $\Mnull$ plus the dark matter halo for one $m_{\chi}$.
  
With these two hypotheses, the likelihood function was maximized to find the best-fit model parameters for each $m_{\chi}$.
In Table \ref{tab:tsvals}, the TS values from each dark matter mass are shown.
Using Wilk's theorem~\cite{wilks1938} with one free parameter, a TS value greater than 20 would hint at the presence of a dark matter halo.
The fact that all of these are less than zero shows that the null (no dark matter) hypothesis is statistically favored.
The next steps are to examine what dark matter cross sections were ruled out with an upper limit calculation.
This is done in Section~\ref{upper_limit}.
Before these upper limits are calculated, some verification can be done to check that the analysis was modeling the data properly.

\begin{table}[!t]
  \centering
  \begin{tabular}{|r|r|}
    \hline
    \textbf{DM Mass}        & \textbf{Halo TS} \\
    \textbf{$m_\chi$} [TeV] &                  \\
    \hline 
    %\input{images/likelihood_analysis/tsvals.tex}
    % taken from /afs/ifh.de/group/cta/scratch/nkelhos/dm_halo_testing/veripy/thesis/analysis/dm_plus_pnt/latextables/tsvals.tex
    % which is generated from the dm_ts values in:
    %   /afs/ifh.de/group/cta/scratch/nkelhos/dm_halo_testing/veripy/thesis/analysis/dm_plus_pnt/logs/info*nfits1000*withpntsrc.json
    % except the 45TeV which is from:
    %   /afs/ifh.de/group/cta/scratch/nkelhos/dm_halo_testing/veripy/thesis/analysis/dm_plus_pnt/refined_analysis/ultrapure_analysis/output/brbbbar.45.00TeV.nfits1000.ctlike.results.json
     4.250 & -0.034 \\
     6.500 & -0.184 \\
    10.000 & -0.457 \\
    14.300 & -0.055 \\
    20.512 & -0.044 \\
    30.400 & -0.009 \\
    45.000 & -0.743 \\
    66.500 & -5.894 \\
    98.000 & -0.026 \\

    \hline 
  \end{tabular}
  \caption[DM Halo TS Values]{
    TS values for each dark matter halo model likelihood ratio maximization fit.
    % think about later, or speed up simulations
  }
  \label{tab:tsvals}
\end{table}
% 100 TeV mass limit?
% From PHYS. REV. D 98, 023016 (2018), 
%   Above 100 TeV, relic density doesn't come out right:
%     K. Griest and M. Kamionkowski, Phys. Rev. Lett. 64, 615 (1990).
%     K. Blum, Y. Cui, and M. Kamionkowski, Phys. Rev. D 92, 023528 (2015).

As there are 948 camera background models in each likelihood analysis, each with a free normalization and spectral index, these are histogrammed as a sanity check in Figure~\ref{fig:param_hist}.
The spectral index distribution is centered on zero because the power law is applied to an existing (non-power-law) background shape.
This means that on average the background models required little spectral hardening or softening.
Because the spectral index is centered on zero, and the spread of spectral indices (\nicetilde0.2) is comparable to the statistical uncertainty on the Crab Nebula spectrum fit (Section~\ref{sec:crab_analysis}, $\gamma=2.31\pm0.17$), this confirms that it is enough to only leave the normalization free.

\begin{figure}[bt]
  \begin{tabular}{ll}
    \includegraphics[scale=0.425]{images/likelihood_analysis/plot_brbbbar_45_00TeV_paramhist_pref.pdf} &
    \includegraphics[scale=0.425]{images/likelihood_analysis/plot_brbbbar_45_00TeV_paramhist_indx.pdf}
  \end{tabular}
  \caption[Histogram of Background Model Parameter Values in the Sgr A* Analysis]{
    Histogram of the two free parameters in each of the 948 camera background models.
    Only the parameters from the $m_\chi\;=\;$\SI{45}{\TeV{}} likelihood fit are shown here.
  }
  \label{fig:param_hist}
\end{figure}
  
To check that the other models are fitting the observed events, we can compare the observed vs modeled counts in different slices.
In Figure~\ref{fig:gc_profile_gal_l}, the observed counts are shown compared to the final, likelihood-optimized modeled counts.
The histogram bins are located on \ang{0.22}-wide slice along the galactic $l$ axis, centered on Sgr A*'s galactic $b$ coordinate.
Yellow histogram bins are the modeled counts from only the camera background models.
Green histogram bins are the total modeled counts from the camera background models, the Galactic Center point source, and the dark matter halo.
Blue points with error bars are the observed counts in each bin, with Poissonian errors.
The observed counts are higher on the left and lower on the right than the modeled histogram.
This is likely due to unaccounted-for elevation effects in the camera background models.
The impact of this elevation effect is tested later in Section~\ref{sec:elevgradient}.

\begin{figure}[t]
  \centering
  \includegraphics[width=0.95\textwidth]{images/likelihood_analysis/plot_brbbbar_45_00TeV_profile_laxis_json.pdf}
  \llap{
    \makebox[14.4cm][l]{ % x position
      \raisebox{7.3cm}{     % y position
        {
          \setlength{\fboxsep }{0pt}
          \setlength{\fboxrule}{1pt}
          \fbox{
            \includegraphics[height=3cm]{images/likelihood_analysis/plot_brbbbar_45_00TeV_profile_laxis_json_minimap_rasterized.png}
          }
        }
      }
    }
  }
  \caption[Galactic Center Profile vs Galactic $l$]{
    Observed vs modeled counts in a slice of the sky parallel to the galactic $l$ axis at $b=-0.046{}^{\circ}$.
    Modeled counts from camera background models are shown in yellow, total modeled counts are shown in green.
    Observed counts with Poissonian errors are shown in blue.
    The bottom plot shows the $\frac{\mathrm{all\:models}}{\mathrm{data}}$ residual as green points.
    The inset plot shows the counts map from Figure~\ref{fig:gc_counts_skymap}, with blue squares showing the profile bin locations.
  }
  \label{fig:gc_profile_gal_l}
\end{figure}

Figure~\ref{fig:gc_profile_gal_b} is similar to Figure~\ref{fig:gc_profile_gal_l}, except it slices through Sgr A* along the galactic $b$ axis.
The feature in which the observed counts are higher on the left and lower on the right is also visible here.
Figure~\ref{fig:gc_profile_energy} shows a similar profile, except it is integrated over a \ang{0.2}$\times$\ang{0.2} galactic ($l$,$b$) square centered on Sgr A*.

\begin{figure}[t]
  \centering
  \includegraphics[width=0.95\textwidth]{images/likelihood_analysis/plot_brbbbar_45_00TeV_profile_baxis_json.pdf}
  \llap{
    \makebox[14.4cm][l]{ % x position
      \raisebox{7.3cm}{   % y position
        {
          \setlength{\fboxsep }{0pt}
          \setlength{\fboxrule}{1pt}
          \fbox{
            \includegraphics[height=3cm]{images/likelihood_analysis/plot_brbbbar_45_00TeV_profile_baxis_json_minimap_rasterized.pdf}
          }
        }
      }
    }
  }
  \caption[Galactic Center Profile vs Galactic $b$]{
    Observed vs modeled counts in a slice of the sky parallel to the galactic $b$ axis at $l=359.944^{\circ}$.
    Modeled counts from camera background models are shown in yellow, total modeled counts are shown in green.
    Observed counts with Poissonian errors are shown in blue.
    The bottom plot shows the $\frac{\mathrm{all\:models}}{\mathrm{data}}$ residual as green points.
    The inset plot shows the counts map from Figure~\ref{fig:gc_counts_skymap}, with blue squares showing the profile bin locations.
  }
  \label{fig:gc_profile_gal_b}
\end{figure}


\begin{figure}[t]
  \centering
  \includegraphics[width=0.95\textwidth]{images/likelihood_analysis/plot_brbbbar_45_00TeV_profile_energy_json.pdf}
  \llap{
    \makebox[6.2cm][l]{ % x position
      \raisebox{5.4cm}{ % y position
        {
          \setlength{\fboxsep }{0pt}
          \setlength{\fboxrule}{1pt}
          \fbox{
            \includegraphics[height=3.6cm]{images/likelihood_analysis/plot_brbbbar_45_00TeV_profile_energy_json_minimap_rasterized.pdf}
          }
        }
      }
    }
  }
  \caption[Galactic Center Profile vs Energy]{
    Observed vs modeled counts in a \ang{0.2}$\times$\ang{0.2} square bin centered on Sgr A* at energies from \SIrange{4}{70}{\TeV{}}.
    Modeled counts from camera background models are shown in yellow, total modeled counts are shown in green.
    Observed counts with Poissonian errors are shown in blue.
    The bottom plot shows the $\frac{\mathrm{all\:models}}{\mathrm{data}}$ residual as green points.
    The inset plot shows the counts map from Figure~\ref{fig:gc_counts_skymap}, with a blue square showing the profile location.
    }
  %}
  \label{fig:gc_profile_energy}
\end{figure}

To check for any poorly modeled areas of the sky, the difference between the observed counts and models can be examined.
The observed counts and models in this Galactic Center analysis are broken up into individual sky-map bins.
Then, each bin's significance (how significantly the models differ from the data) is calculated with Equation~\ref{eqn:resmap_signif} and shown in Figure~\ref{fig:gc_resmap}.
In this plot, the deficiency of the radially-symmetric camera background models is apparent.
Lighter/darker areas indicate places where there were more/fewer observed counts than the best-fit models predicted.
The darker upper-left and lighter lower-right areas are due to the gradient in the atmospheric air mass.
This creates a gradient in the background that is not radially symmetric, which the radially-symmetric background models cannot properly fit.
Several darker bins in a ring-like pattern around the edge of the field of view may be due to how the background models behave at the edge of the camera (see Figure~\ref{fig:background_radial}), or how events near the edge of the camera may be poorly reconstructed at low telescope elevations {\color{red}(does this sentence sounds ok??)}.

{\color{red}(in above paragraph, do you have any cut on edge bins (during reconstruction)? what is the field of view you are considering in your analysis? how does Figure~\ref{fig:gc_resmap_sighist_coarse} look like without those edge bins? is it the expected N(0,1) normal distribution? -gernot,  in significance skymap, add sentence about how there may be a contribution from edge bins (the dark ring gernot mentioned??))}

\begin{figure}[ht]
  \centering
  \includegraphics[width=0.95\textwidth]{images/likelihood_analysis/plot_brbbbar_45_00TeV_resmap_coarse.pdf}
  \caption[Galactic Center Residual Map]
  {
    Significance of each sky-map bin's residual counts.
    See Equation~\ref{eqn:resmap_signif} for details.
  }
  \label{fig:gc_resmap}
\end{figure}

The distribution of the pixel significances from Figure~\ref{fig:gc_resmap} is also shown in gray in Figure~\ref{fig:gc_resmap_sighist_coarse}, along with a set of 10000 simulated distributions in red.
If the source and background models were perfectly known, the gray bars would overlap the red simulations in most of the bins, and the residual will overlap 1.
Because only four of the residual bins are consistent with 1, differences are observed between the models and the data.
Specifically, there are more negatively-significant and positively-significant data pixels than expected from the models, hinting that one or more features in the Galactic Center data are not being modeled properly.
This is discussed further in Section~\ref{sec:elevgradient}.
The peaked bin around $\sigma=0$ is likely due to the bins at the edge of the Galactic Center data, where the residual approaches zero due to the negligible (and poorly sampled) VERITAS sensitivity at the edge of the camera {\color{red}(my first instinct is to ask you to check this claim because it should be very quick.  I hope it won't be the instinct of the reviewers...   It seems to me a little too big to come from just the edge of the camera.  Are you sure you are not plotting also the edges of the sky map where you have no entries?? -orel) (either cut the bins that have 0 counts, or check where the 0-significance bins are actually coming from to get rid of the 'likely' bit) }.
{\color{red}(see plots, its not conclusive, soften word 'likely' to 'may' or something??)}

\begin{figure}[ht]
  \centering
  \includegraphics[width=0.95\textwidth]{images/likelihood_analysis/plot_brbbbar_45_00TeV_resmap_coarse_pdf_json_hist.pdf}
  \caption[Galactic Center Coarse Residual Histogram]
  {
    Histogram of the residual sky-map bin significances from Figure~\ref{fig:gc_resmap}, calculated with Equation~\ref{eqn:resmap_signif}.
    In the top plot, the gray histogram bars indicate the data’s significance distribution, with gray Poissonian error bars.
     The red lines and bars indicate the range of simulated significances, centered on the mean, and extending up and down by one standard deviation.
    The bottom plot shows the $\frac{\textrm{simulated}}{\textrm{data}}$ residual.
    Only sky-map bins within Sgr A*'s region of interest are histogrammed.
  }
  \label{fig:gc_resmap_sighist_coarse}
\end{figure}

The Galactic Center point source spectrum can be plotted as a check of the likelihood fitting.
This spectrum is shown in Figure~\ref{fig:gc_pntsrc_spectrum}.
This point source was detected with a test statistic of 224.25 (with one degree of freedom), with a normalization of $(4.764\pm0.402)\times 10^{-18} \frac{\textrm{photons}}{\textrm{cm}^2\;\textrm{s}\;\textrm{MeV}}$.
A statistical uncertainty band is shown as a very thin red region, and a combined statistical plus systematic uncertainty is shown as a slightly thicker red band.
The systematic uncertainty comes from showers of different energies forming similar images in the VERITAS telescopes, leading to a \nicetilde20\% systematic uncertainty on the flux.
The difference between the two spectra is likely due to the previous VERITAS being made from a combined fit with data from another telescope.

\begin{figure}[ht]
  \centering
  \includegraphics[width=0.95\textwidth]{images/likelihood_analysis/plot_brbbbar_45_00TeV_spectrum.pdf}
  \caption[Galactic Center Point Source Spectrum]
  {
    The spectrum of the fitted Galactic Center point source is shown in red, with statistical and combined statistical+systematic uncertainty bands.
    Only the normalization was free in the likelihood fit.
    The original VERITAS spectrum is shown in purple, from Ref.~\cite{VeritasGCRidge2015}.
    Note that the original VERITAS spectrum is part of a combined fit with another gamma-ray telescope.
  }
  \label{fig:gc_pntsrc_spectrum}
\end{figure}

\FloatBarrier

\section{Upper Limit}\label{upper_limit}

Because no dark matter signal was detected, the next step is to calculate what parts of the parameter space are ruled out.
This is done by calculating an upper limit on the WIMP's velocity-averaged cross section $\vacs$.
From the dark matter flux 

\begin{equation}\label{eqn:dmflux2}
  \frac{ d\Phi }{ dE d \Omega } = \frac{ \left \langle \sigma v \right \rangle }{8 \pi m_\chi^2} \frac{dN_{\gamma}}{dE} \int \rho^2 dl \,,
\end{equation}
(see Equation~\ref{eqn:dmflux} for details), an upper limit on the observed gamma-ray flux can be derived.
When all other variables in Equation~\ref{eqn:dmflux2} are fixed, a larger $\vacs$ always results in a larger gamma-ray flux.
This means that the flux and cross section can both be replaced by their upper limits ($x \rightarrow x \left |_{\textrm{ul}}$), as in

\begin{equation}\label{eqn:ulim}
  \frac{d\phi}{dE d\Omega} \left. \right |_{\textrm{ul}} = \frac{ \left \langle \sigma v \right \rangle \left | \right _{\textrm{ul}} }{8\pi m_{\chi}^{2}} \frac{dN_{\gamma}}{dE} \int \rho^{2} \; dl \,.
\end{equation}

To calculate the upper limit, the parameter $\vacs$ is fixed and the maximum likelihood is found.
Then, $\vacs$ is fixed to a slightly larger value, and a new maximum likelihood is found.
Because no dark matter signal was found, increasing $\vacs$ increases the dark matter halo's flux, which decreases the maximum likelihood.

This procedure of increasing a parameter and rerunning the likelihood forms a likelihood profile.
Naively, this profile is shaped like a Gaussian distribution, as shown in Figure~\ref{fig:ulimitdiag}.
This Gaussian shape can be integrated to find the one-sided limit that is above the true parameter value with a specified confidence level.
Specifically, this integration determines how far down the likelihood must decrease to reach the upper limit.
This likelihood decrease is called $\Delta\LogLike$, and is calculated via

\begin{equation}\label{eqn:deltaloglike}
  \Delta\LogLike = \left ( \textrm{erf}^{-1} \left ( \alpha \right ) \right )^2 \,,
\end{equation}
where $\alpha$ is the desired confidence level as a fraction of 1 (i.e. 95\% : $\alpha=0.95$)~\cite{wilks1938,cash1979}.
The $\textrm{erf}^{-1}$ is the inverse error function, an integration of a Gaussian function.
Therefore, an upper limit on $\vacs$ can then be found by increasing $\vacs$ to the point where the likelihood has decreased by $\Delta\LogLike$.


%This procedure of increasing $\vacs$ and rerunning the likelihood maximization is 
%repeated until the likelihood has decreased by a fixed amount.
%Because the likelihood profile (the maximum likelihood as one parameter is varied) naively follows a Gaussian distribution as in Figure~\ref{fig:ulimitdiag}, this can be exploited.
%
%To find the value of $\vacs$ that is above 95\% of the outcomes, a Gaussian distribution can be integrated to find how far the likelihood decreases.
%%This factor is chosen because the likelihood naively follows a Gaussian distribution as any parameter is varied through its best-fit value.
%This likelihood decrease is $\Delta{}\LogLike$, and is calculated by integrating a Gaussian probability distribution, resulting in

%For that Gaussian distribution, the factor $\Delta{}\LogLike$ is how much the likelihood decreases when integrating from a one-sided limit.
%This is shown in Figure~\ref{fig:ulimitdiag}.
% some discussion on likelihood upper limits here:
% https://fermi.gsfc.nasa.gov/science/mtgs/summerschool/2013/week1/ML_intro.pdf
% maybe also this rolke paper: Rolke, et al., NIM A, 551, 493 (2005)
% see bookmark in Geln Cowan Statistical Data Analysis

{\color{red}(
  Honestly, this explaination is so convoluted and so different than what I am familiar with that I can't say if it's correct.
  I just spent an hour reading various chapters in the owan book and I can't find what you do in it.
  It seems to me like a mix of two different methods, but I am not sure.
  I figure you are explaining the procedure as it performed in ctools? are you 100\% sure about it?
  Technically the value of DL should depend on the number of free parameters as well, so I don't know about Equation~\ref{eqn:deltaloglike}, but if it is what is used in ctools I assume it is correct.
  Either way you do not calculate the upper limit yourself right? ctools does it for you?
  Without going into ctools code and see what they do I can't say much, so I suggest you ask gernot to read these paragraphs and if he gives the ok, leave it as it is.  ?? --orel
)}

\begin{figure}[bt]
  \centering
  \includegraphics[width=0.45\textwidth]{images/ulimit_diagram/ulimdiagram.pdf}
  \caption[Upper Limit Calculation]{
    An example likelihood profile as the $\vacs$ is varied (blue line).
    The procedure starts at parameter $\vacs_{\mathrm{start}}$, and $\vacs$ is increased until the likelihood has decreased by $\Delta{}\LogLike$.
    {\color{red}(
      I think you are mixing two methods in this plot.
      One is wher eyou find the value of the parameter which decreases L by some value DL and the other is where you calculated the p value based on the observed value of your parameter.
      Again, I am so confused at the moment (and tired) that I am not sure, but please verify before you leave this plot as it is.
      Talk to Gernot, I am sure he can help with it. ?? --orel
    )}
  }
  \label{fig:ulimitdiag}
\end{figure}



From the nine WIMP masses tested in Section~\ref{sec:dmlike}, the calculated upper limits are shown in Figure~\ref{fig:ulim} as dark blue points with error bars.
Several other upper limits from previous studies are also shown for comparison.
The orange upper limit is from 216 hours of VERITAS observations of several dwarf spheroidal galaxies, stacked into a single likelihood analysis~\cite{verdsphul}.
The dark green upper limit is from 254 hours of H.E.S.S. observations of the Galactic Center~\cite{hessgcul}.
The red upper limit is from a combined Fermi-MAGIC likelihood analysis using observations from several dwarf satellite galaxies~\cite{fermagicul}.
Fermi and MAGIC spent a total of 6 years and 158 hours observing these galaxies, respectively.
The light blue dashed line is the relic cross section~\cite{updatedWIMPRelicCrossSection}.

It is important to note that the width of a dark matter signal is not known at this time.
As such, the nine upper limits provided by this work do not include a width, so only upper limits at those masses are valid.
This study can be expanded in the future to account for the width of a WIMP signal to provide upper limits over a larger range of dark matter masses.

\begin{figure}[tb]
  \centering
  \includegraphics[width=0.85\textwidth]{images/ulimit/ulimit.pdf}
  \caption[Dark Matter Upper Limit]{
    Dark matter upper limits on velocity-averaged cross sections.
    The results from this work are in dark blue.
    Orange is from a VERITAS~\cite{veritas_dm_limit} analysis of dwarf spheroidal galaxies.
    Dark green is from a H.E.S.S.~\cite{hess_dm_limit} analysis of the Galactic Center.
    Purple is from a HAWC~\cite{hawc_dm_limit} analysis of dwarf spheroidal galaxies.
    Red is from a joint Fermi-MAGIC~\cite{fermagicul} analysis.
    Bright green is from Planck measurements of the CMB~\cite{planck_dm_limit}.
    The light blue dashed line is the relic cross section~\cite{updatedWIMPRelicCrossSection}.
    Data values from this work's upper limits are in Appendix Table \ref{tab:ulimvals}.
  }
  \label{fig:ulim}
\end{figure}

These limits were calculated using a simple WIMP model, one that does not include the effect of Sommerfeld enhancement~\cite{sommerfeld}.
If Sommerfeld enhancement were included in the dark matter halo model, stronger limits would naively be expected.

Around the dark blue upper limit points from this work, systematic uncertainty error bars are also shown.
For VERITAS, uncertainties in the energy reconstruction contribute a \nicetilde{}20\% systematic uncertainty on the flux.
While no dark matter was detected in this thesis, the upper limit calculation is influenced by this systematic uncertainty.
This can be envisioned as a dark matter signal that was brighter than the upper limit in Figure~\ref{fig:ulim}, but then fluctuated downwards by 20\% of its flux.
Conversely, a dim dark matter signal below the upper limit could be enhanced by 20\% instead.
These uncertainties are reflected by the blue error bars.

\FloatBarrier

\section{Impact of Elevation Gradient}\label{sec:elevgradient}

The residual map in Figure~\ref{fig:gc_resmap} has a visible gradient, going from the lower right to the upper left.
This means that the radial camera background models are not accurately fitting the shape of the observed events.
The direction of the residual gradient aligns with the elevation gradient in the camera, visible in Figure \ref{fig:background_grid}.
To check that this elevation gradient has a negligible effect on the upper limit result in Figure~\ref{fig:ulim}, the likelihood analysis was redone with an elevation gradient.

As this analysis is limited to the energy range \SIrange{4}{70}{\TeV{}}, the elevation gradient will increase the number of events detected at the bottom of the camera, due to the thicker atmosphere providing increased interaction mass.
And, since the Galactic Center is visible almost directly south at an azimuth of \ang{193}, the elevation axis is almost parallel to the declination axis.
A gradient of 5\%/degree was chosen due to the relative differences between the total modeled counts and the observed counts in Figures \ref{fig:gc_profile_gal_l} and \ref{fig:gc_profile_gal_b}.
% diffuse simulations produced here:
%   /afs/ifh.de/group/cta/scratch/nkelhos/veripyctools/recon.sim.reproduce_crescent/sim.py
% gradient chosen using
% /afs/ifh.de/group/cta/scratch/nkelhos/dm_halo_testing/veripy/thesis/analysis/elevation_gradient/measure_existing_gradient/plot.py
% when run, pgrad slope indicates event efficiency changes at rate of 1.5% per degree of elevation (at 4 TeV)
%   also look at plot energy_elevation_gradients.png to see diffuse simulated gradient
% looking at counts in figure suggests a higher value, around 10%/deg?
% I think I picked halfway between these two numbers, 5%/deg
An example background is shown in Figure~\ref{fig:bkg_flatvsgrad}, before and after the gradient is applied.
While faint, the difference between the two can be seen in the central dark-red ring.
In the left, the ring is symmetric, while in the right, the ring is darker at the bottom than at the top.

\begin{figure}[tb]
  \centering
  \includegraphics[width=0.98\textwidth]{images/background_gradient/gradientcomparison.pdf}
  \caption[Background Gradient Comparison]{
    The background models for run 68348, minutes 16-24.
    Left is the original radial background, centered \ang{0.5} north of the Galactic Center.
    Right is the same background, after applying the 5\%/degree gradient along the elevation axis.
  }
  \label{fig:bkg_flatvsgrad}
\end{figure}

After these were applied to all background models, the likelihood analysis was re-run for the \SI{10}{\TeV{}} WIMP mass.
This can be considered representative of WIMP masses above \SI{10}{\TeV}, as the background does not strongly change in that range (see Figure~\ref{fig:background_grid}).
The resulting upper limit with the gradient was only 0.1\% different from the upper limit with radial background models.
This indicates the upper limits are not very sensitive to the background models.
% on warp-zeuthen.desy.de:
% cd $VERIPY/thesis/analysis/dm_plus_pnt/
%
% no gradient:
% $ grep "crosssection_ul" logs/statistics.brbbbar.10.0TeV.nfits1000.withpntsrc.txt 
% [1.6279762902243517e-24, 'cm^3/s']
%
% with gradient:
% $ grep "crosssection_ul" with_elev_gradient/logs/statistics.brbbbar.10.0TeV.nfits1000.withpntsrc.txt 
% [1.6280879530415155e-24, 'cm^3/s']
This likely stems from the fact that the test statistic is a ratio of the likelihoods of two model groups, both of which contain the background models.
Thus, the likelihoods from both model groups changed by a similar amount, and in the same direction, partially cancelling out.



