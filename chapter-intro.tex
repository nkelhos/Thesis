\cleartooddpage[\thispagestyle{empty}]
\chapter{Introduction}


\section{Dark Matter}

\section{Galactic Center}

\section{Gamma-Ray Astronomy}

\section{Science Objectives}

\section{Gamma-Ray Showers}
\section{Gamma Ray Detectors}

Veritas

Veritas is a gamma-ray observatory.
This observatory consists of 4 telescopes, spread out over an area of 50 meters.
Each telescope consists of a 12m dish of 498 mirrors, with a customized 498 pixel single-photon timing camera at the end of a 7 meter boom.

Gamma rays are reconstructed and observed through a multi-stage process.
First, gamma rays are emitted from an astrophysical source, outside the solar system.
Then, these gamma rays travel to Earth and interact with molecules of Earth's atmosphere.
When they interact, they pair-produce into an electron-positron pair, where each has an energy of half the parent gamma ray.
The positron will then annihilate with an atmospheric electron, producing 2 more gamma rays, while the electron will shed some energy through ionization and by emitting chernekov photons.

This process will repeat over and over, creating a particle shower, until the electrons have ~80MeV of kinetic energy, at which point ionization will dominate the energy loss, and the shower will stop producing cherenkov light.

Veritas telescopes are designed to detect the cherenkov photons emitted from these particle showers, timing their arrival in each pixel to within 1 nanosecond.
The shower images produces by multiple telescopes can then be used to triangulate the original gamma ray's point of orgiin, whereas the size of the shower can be used to determine the gamma ray's energy.


Effective Area
For the calculation of the flux, the detection area must be found.
This depends primarily on each gamma ray's energy, as well as its detected position in the camera.

