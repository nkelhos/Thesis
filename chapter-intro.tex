\cleartooddpage[\thispagestyle{empty}]
\chapter{Introduction}

How much mass is contained within a given volume of space?
A Galaxy?
A Cluster of Galaxies?
The observable universe?

These questions have been asked repeatedly since the 1930s, but different measurements produce conflicting answers.
Observing cosmological and gravitational features results in a larger mass, while observing the quantity of light produced results in a smaller mass.
This implies that there is missing mass, missing matter, unaccounted for by the existing model of physics and our universe.
As this matter prefers to only interact gravitationally and seemingly ignores the electromagnetic spectrum, it has earned the (in)conspicuous title of Dark Matter.

\section{Motivation}
This thesis will demonstrate that, by analyzing gamma-rays from the galactic center, a search for a Dark Matter Halo can be conducted.
If Dark Matter annihliation is able to produce a detectable flux, the density profile of the Dark Matter halo can be measured.
If instead the flux is below the limits of our telescope, then an upper limit can be placed on the cross-section of the Dark Matter Particle.

\section{Dark Matter}

% WMAP: 4.6% energy from baryonic matter, 24% energy from cold dark matter
From measuring the cosmic microwave background, WMAP satellite found that Dark Matter makes up 24\% of the energy of the universe\cite{pdg_2012}, thus understanding the nature of Dark Matter is fundamental to understanding our universe.
It is currently believed that Dark Matter is new particle, unknown to the standard model of particle physics.
This particle is predicted to interact primarily through Weak interactions(??), thus it is referred to as a Weakly Interacting Massive Particle, or WIMP.
From both cosmology and particle physics, the WIMP is predicted to have a mass in the range of GeV to TeV, and a cross-section around ?? $gcm^{-2}$.
From Weak interactions, WIMPs may directly annihilate or decay into gamma rays, or they may first produce quarks or leptons, which would then produce gamma rays.
Many of these gamma rays would then have energies similar to the original WIMP mass, around 1 TeV.
This potential for WIMP Dark Matter to produce TeV gamma rays makes it an attractive science target for gamma-ray observatories like VERITAS.

Then, the question becomes, where should we point our gamma-ray observatories?
From gravitational and optical measurements, it is well documented (cite??) that halos of Dark Matter augment almost all dwarf galaxy, galaxy, and galaxy cluster gravitational wells.
Dwarf galaxies tend to have fewer background gamma-ray sources, but also have lower quantities of Dark Matter, making them weaker sources of dark matter gamma rays.
Galaxy Clusters, while more massive (and thus more emissive), have a non-negligable redshift, introducing more model parameters and complicating any analysis (??).
The galactic center, on the other hand, possesses more Dark Matter mass at higher densities than any dwarf galaxy, while also being closer than any dwarf galaxy(??), making it an excellent target for gamma-ray Dark Matter searches.


\section{Galactic Center and Gamma Rays}

At the center of our galaxy there is believed to be a supermassive black hole, with a mass of around $~4*10^6m_{\odot}$(cite??).
As Dark Matter appears to accompany most Baryonic gravitational wells(??), it is expected that there is a halo of dark matter particles the galactic center.
Studying this halo is difficult, however, as our galaxy's large graviational well has accreted a large amount of dust, as well as there being a large number of stars and supernova remntants nearby.

\begin{figure}[h]
  \begin{center}
    \includegraphics[width=0.85\textwidth]{images/multiwavelength_galaxy.eps}
    \caption[Multiwavelength Milky Way]{The plane of our galaxy, viewed in different wavelengths.  The analysis described in this thesis examines the central \nicetilde5\degree\cite{milky_way}}
  \end{center}
\end{figure}

The dust, supernova remnants, and the galactic center itself all emit gamma rays, making detection of a dark matter halo difficult.
The dust provides a collision target for high-energy protons from other galaxies, which produce $\pi_0$'s, which then decay into gamma rays.
These dust-induced gamma rays are collectivly referred to as Diffuse Emission, and appear as a disk of gamma rays along the galactic plane.
The supernova remnants accelerate electrons and protons outwards, which then shock into the surrounding dust and gas, producing gamma rays.
The black hole at the galactic center also produces gamma rays in a sharp point (relative to the other sources), though the mechanism by which it does this is not well understood(??).

These effects all obscure the target of this analysis, the gamma ray emission from a dark matter halo.
This spherically-symmetric halo of gamma rays would surround the galactic center, decreasing in intensity the further from the center.
The halo's spectrum would also be different from the surrounding diffuse emission.

Once these gamma rays have been emitted, they would travel to Earth, where humans can detect them.
This is done by using a telescope to observe the particle shower that occurs when each gamma ray hits the Earth's atmosphere.
For this analysis, the VERITAS telescope has been observing gamma rays from the galactic center for several years.

\section{Likelihood Analysis}
Once a set of gamma rays are detected, a likelihood analysis can be used to model each distict component of the gamma ray emission.
This analysis is performed by calculating the likelihood of two different scenarios (usually referred to as hypotheses).
The first scenario is that there is a dark matter halo alongside the galactic center's point emisssion and the diffuse emission.
The second scenario describes the gamma-ray emission from just the galactic center's point emission and the diffuse emission, with no dark matter.
By calculating the ratio of the likelihoods of these two scenarios, the question 'Does a dark matter halo exist around the galactic center?' can be answered.
If the answer is no, then an upper limit can be calculated for the cross-section of the dark matter particle, which may eliminate some of the dark matter theory parameter space.


