\cleartooddpage[\thispagestyle{empty}]
\chapter{Introduction}

How much mass is contained within a given volume of space?

A Galaxy?

A Cluster of Galaxies?

The observable universe?

The question 'How Much Mass is in an Expanse of Space' (rephrase??) has been asked repeatedly since the 1930s, but different methods of measuring the answer always disagree.
Observing cosmological and gravitational features results in a larger answer, while observing the quantity of light produced results in a smaller one.
This implies that there is missing mass, missing matter, unaccounted for by existing physics.
As this matter prefers to only interact gravitationally and ignores the electromagnetic spectrum, it has earned the (in)conspicuous title of Dark Matter.

\section{Motivation}
This thesis will demonstrate that, by analyzing gamma-rays from the galactic center, a search for a Dark Matter Halo can be conducted.
If Dark Matter annihliation is able to produce a detectable flux, the density profile of the Dark Matter halo can be measured.
If the flux is below the limits of our telescope, then an upper limit can be placed on the mass and cross-section of the Dark Matter Particle.

\section{Dark Matter}

As Dark Matter is predicted to be ?? \% of the mass of the universe and \% of the energy of the universe, it is very pervasive.
However, only interacting gravitationally means it is difficult to detect.
There are several that have been developed that can measure the amount of dark matter in a region of space.



\section{Galactic Center}

At the center of every galaxy there is believed to be a supermassive black hole, with a mass of millions of suns.
The Milky Way's own supermassive black hole is measured to be $~4*10^6$ solar masses.
As Dark Matter and Baryonic matter coincide most places in the universe (??), it is expected that there is a dark matter halo around the galactic center.
Studying this halo is difficult, however, as the black hole's graviational well has accreted a large amount of material, as well as there being a large number of stars nearby.

?? Picture of GC in radio

In the galactic center, there is the the black hole itself, whose gamma-ray emission is not fully understood.
There is also a disk of dust, stretching in a line across Earth's field of view.
There are several supernova remnants, slowly-expanding shells of energetic particles, shocking into the ambient dust, as well as many other dust clouds whose origin is not well understood.


\section{Gamma-Ray Astronomy}
Humans have been attempting to detect gamma rays for several decades now, and they're starting to get the hang of it.
Usually this is done by chosing a volume of solid, liquid, or gas matter, setting up sensitive particle detectors, waiting for gamma rays to strike the volume, and observing the particles that are emitted by the strike.
As even the the massive emitted particles are still travelling at a large fraction of light-speed, very high-speed electronics are needed to record the signals.


\section{Gamma-Ray Showers}
Gamma rays are very high energy photons.
When these photons strike the Earth, the majority spallate off of atoms in the atmosphere.
This spallation consumes the photo to produce an electron-positron pair, which then each strike other particles in the atmosphere.
The electrons tend to strike other atomic electrons, knocking them free, while the positrons will annihilate with other atomic electrons, producing a pair of photons.
These particles continue to strike and divide, cascading in an oval-shaped shower of energetic particles along the original direction of the primary gamma ray.
Eventually, as the number of particles in each successive generation is more numerous than the previous, the average energy of each particle goes down.
Once the electrons fall to around 80 MeV ??, then they lose energy rapidly to ionization of atmospheric atoms, and are no longer able to create new generations of particles.
Throughout this shower, the charged electrons and positrons are passing through the atmosphere at a speed very close to $c_{vaccume}$.
As the speed of light in the atmosphere is less than the relativistic speeds of the particles, the charged particles will travel through the atmosphere, outracing the effects of their charge on nearby atmospheric molecules.
This outracing creates a wave of polarziation in the atmospheric molecules, resulting in the emission of lower-energy (visible and ultraviolet) photons known as Cherenkov photons, at a small angle relative to the charged particles' direction of travel.

\section{Gamma Ray Observatories}
Modern gamma-ray observatories usually function by detecting the particles created by the gamma ray as it strikes some known interaction mass.
The interaction mass can be as small as 1 $m^3$ or as large as the atmosphere on earth.
Veritas functions by watching for gamma-ray showers, by using Earth's atmosphere as the interaction mass.
Gamma-ray showers produce Cherenkov photons in a very short amount of time, typically on the order of ~10s of nanoseconds.
By constructing a specialized camera that counts the Cherenkov photons from the sky at ~1ns intervals, the gamma ray's shower can be imaged.
By combining multiple synchronized cameras spaced at roughly ~100m apart, multiple images can be taken of the same showers.
As the showers are highly directional along the original gamma-ray's trajectory, the images then point backwards to the original gamma-ray's point-of-origin in the sky.
By running a large amount of simulations, a database of shower sizes and positions can be built, allowing for the energy of each gamma ray to be reconstructed.

\section{Likelihood Analysis}

A likelihood ratio test fits two separate functions to datapoints, and then calculates the probability that one function is the more likely fit (called the Likelihood Ratio).
Using Wilk's theorem??, the Likelihood Ratio can then be converted to a significance.
The first function is called the Null Hypothesis, and the second function is the Alternative Hypothesis.
In the context of astrophysical searches, each hypothesis predicts how many gamma rays will be at each point in the space/energy/time dataspace.

