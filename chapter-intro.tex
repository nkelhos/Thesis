\cleartooddpage[\thispagestyle{empty}]
\chapter{Introduction}

How much mass is contained within a given volume of space?
A Galaxy?
A Cluster of Galaxies?
The observable universe?

Questions of 'How Much Mass is in a Region of Space' have been asked since the 1930s, but different methods of measuring the answer always disagree.
Observing cosmological and gravitational features results in a large answer, while observing the quantity of light produced results in a smaller one.
This implies that there is missing mass, missing matter, unaccounted for by our current models of the universe.
As this matter prefers to only interact gravitationally, without interacting with the electromagnetic spectrum, it has earned the (in)conspicuous title of Dark Matter.


\section{Dark Matter}

From cosmological measurements, Dark Matter consists of a large portion of the universe's mass and energy budget.
By energy, Dark Matter accounts for roughly 27\% (??) of the known universe, with another \~2\% (??) making up baryonic matter, while the remaining 70\% (??) is made up by Dark Energy.
By mass, this means that 95\% (??) of the universe's mass is bound up in Dark Matter, while the remaining 10\%



The primary candidate for the particle that constitutes Dark Matter is a Weakly Interacting Massive Particle, or WIMP.
WIMPs 

Measurements of Dark Matter typically fall into two categories, as stated in the p

Observations of Dark Matter
From observations
Dark Matter 

These disparities vary in scale, from satellite galaxies (??m) to cosmological scales spanning the entirety of the observable universe.




\section{Galactic Center}

At the center of every galaxy there is a supermassive black hole, with a mass of millions of suns.
The Milky Way's own supermassive black hole is measured to be about $4*10^6$ solar masses.

\section{Gamma-Ray Astronomy}
Humans have been attempting to detect gamma rays for several decades now, and they're starting to get the hang of it.
Usually this is done by chosing a volume of solid, liquid, or gas matter, setting up sensitive particle detectors, waiting for gamma rays to strike the mass, and observing the particles that are emitted by the strike.
As even the the massive emitted particles are still travelling at a large fraction of light-speed, very high-speed electronics are needed to record the signals.

\section{Science Objectives}
The goal of this thesis is to demonstrate that, by analyzing gamma-rays from the galactic center, a search for a Dark Matter Halo can be conducted.
If a halo is discovered, it would be a huge discovery.
If a halo is not discovered, this allows for upper limits to be placed on the halo's flux.

\section{Gamma-Ray Showers}
Gamma rays are very high energy photons.
When these photons strike the Earth, the majority spallate off of atoms in the atmosphere.
This spallation consumes the photo to produce an electron-positron pair, which then each strike other particles in the atmosphere.
The electrons tend to strike other atomic electrons, knocking them free, while the positrons will annihilate with other atomic electrons, producing a pair of photons.
These particles continue to strike and divide, cascading in an oval-shaped shower of energetic particles along the original direction of the primary gamma ray.
Eventually, as the number of particles in each successive generation is more numerous than the previous, the average energy of each particle goes down.
Once the electrons fall to around 80 MeV ??, then they lose energy rapidly to ionization of atmospheric atoms, and are no longer able to create new generations of particles.
Throughout this shower, the charged electrons and positrons are passing through the atmosphere at a speed very close to $c_{vaccume}$.
As the speed of light in the atmosphere is less than the relativistic speeds of the particles, the charged particles will travel through the atmosphere, outracing the effects of their charge on nearby atmospheric molecules.
This outracing creates a wave of polarziation in the atmospheric molecules, resulting in the emission of lower-energy (visible and ultraviolet) photons known as Cherenkov photons, at a small angle relative to the charged particles' direction of travel.

\section{Gamma Ray Observatories}
Modern gamma-ray observatories usually function by detecting the particles created by the gamma ray as it strikes some known interaction mass.
The interaction mass can be as small as 1 $m^3$ or as large as the atmosphere on earth.
Veritas functions by watching for gamma-ray showers, by using Earth's atmosphere as the interaction mass.
Gamma-ray showers produce Cherenkov photons in a very short amount of time, typically on the order of ~10s of nanoseconds.
By constructing a specialized camera that counts the Cherenkov photons from the sky at ~1ns intervals, the gamma ray's shower can be imaged.
By combining multiple synchronized cameras spaced at roughly ~100m apart, multiple images can be taken of the same showers.
As the showers are highly directional along the original gamma-ray's trajectory, the images then point backwards to the original gamma-ray's point-of-origin in the sky.
By running a large amount of simulations, a database of shower sizes and positions can be built, allowing for the energy of each gamma ray to be reconstructed.


\section{old}
Veritas

Veritas is a gamma-ray observatory.
This observatory consists of 4 telescopes, spread out over an area of 50 meters.
Each telescope consists of a 12m dish of 498 mirrors, with a customized 498 pixel single-photon timing camera at the end of a 7 meter boom.

Gamma rays are reconstructed and observed through a multi-stage process.
First, gamma rays are emitted from an astrophysical source, outside the solar system.
Then, these gamma rays travel to Earth and interact with molecules of Earth's atmosphere.
When they interact, they pair-produce into an electron-positron pair, where each has an energy of half the parent gamma ray.
The positron will then annihilate with an atmospheric electron, producing 2 more gamma rays, while the electron will shed some energy through ionization and by emitting chernekov photons.

This process will repeat over and over, creating a particle shower, until the electrons have ~80MeV of kinetic energy, at which point ionization will dominate the energy loss, and the shower will stop producing cherenkov light.

Veritas telescopes are designed to detect the cherenkov photons emitted from these particle showers, timing their arrival in each pixel to within 1 nanosecond.
The shower images produces by multiple telescopes can then be used to triangulate the original gamma ray's point of orgiin, whereas the size of the shower can be used to determine the gamma ray's energy.


Effective Area
For the calculation of the flux, the detection area must be found.
This depends primarily on each gamma ray's energy, as well as its detected position in the camera.

