\cleartooddpage[\thispagestyle{empty}]
\chapter{Conclusion}

VERITAS gamma-ray data was used in a new likelihood ratio test to measure the properties of astrophysical gamma-ray sources.
This method is also unique in that it folded the VERITAS PSF, Effective Areas, and Energy Dispersion into the studied models.
This likelihood test was verified with a well-studied source, the Crab Nebula.
Using 9 hours of Crab Nebula observations, this test found that the best fit flux spectral index and normalization were found to be consistant with existing analysis methods.

Having verified this new analysis method, a search for Dark Matter was conducted.
108 hours of gamma-ray observations of the Galactic Center were analyzed with a likelihood ratio test.
This test searched for a halo of WIMP dark matter particles that were annihilating into $b\bar{b}$ quarks, which then cascaded into gamma rays.
The tested halo follows a simple Einasto halo profile, though it does not have a flat density core.
Nine different WIMP masses were tested with this search.
The VERITAS PSF, Effective Areas, and Energy Dispersion were also folded into the WIMP halo models.
As some new behavior was discovered in the background models at the low energies (\SIrange{1.5}{4}{TeV}), only events with a reconstructed energy of \SIrange{4}{70}{TeV} were used in this analysis.
As shown in Table~\ref{tab:tsvals}, it was found that no significant Dark Matter signal was detected at any of the WIMP masses.

To examine what WIMP candidates have been ruled out by this search, an upper limit was calculated for each of the nine WIMP masses.
These upper limits are shown in Figure~\ref{fig:ulim}, where new limits are able to be placed at WIMP masses above \SI{70}{TeV}.
This likelihood ratio test also offers improved sensitivity over the standard On-Off analysis method, providing stronger limits with fewer hours of observation.

To look for areas in the skymap where the models are not optimally fitting the events, a residual skymap is made.
As the halos were not detected, only the camera background models and the Galctic Center point source contributed to the residual skymap.
In this residual skymap, a strong gradient is noted along the observatory's elevation axis.
This gradient is due to the camera background models not properly handling the atmospheric gradient, which at \ang{30} elevation is approximately \nicetilde{}20\% different between the top and bottom of the VERITAS field of view.
By adding a simple elevation gradient to the background models, the sensitivity of the Dark Matter upper limit to the background modeling was tested.
It was found that for a moderate gradient of 5\%/\degree{}, the upper limit at one $m_{\chi}$ only moved by 0.1\%, indicating it is not strongly dependent on the background shape.

There are several improvements which can be made to this analysis method.
The background models created for this thesis indicate the proton background varies with respect to event energy and position in the camera, whereas before this point, only a radial dependence was noted.
These background models can be improved to account for these new dependencies, which would allow for the inclusion of low energy events from \SI{1.5}{4}{TeV}.
This additional energy range would allow for the inclusion of \nicetilde{}40\% more events, improving Dark Matter searches and upper limits.
However, the causes of the variations in camera sensitivity are not well understood, and need to be explored further.
These variations are at least partly due to the atmospheric gradient.
As gamma-ray showers and proton showers behave similarly in the atmosphere, it may also mean that the gamma-ray effective areas and energy dispersions also depend on the (x,y) position within the camera, instead of just the camera radius.
More studies would need to be done to explore this.

VERITAS gamma-ray data was used in a likelihood ratio test, to conduct a basic search for Dark Matter.
No signal was detected, but for cuspy WIMP halos, a new region of the $ \left ( m_{\chi}, \left \langle \sigma v \right \rangle \right )$ parameter space has been eliminated.

