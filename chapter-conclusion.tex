\cleartooddpage[\thispagestyle{empty}]
\chapter{Conclusion}

There is strong evidence that Dark Matter is a WIMP particle.
This evidence includes rotational measurements of dwarf galaxies, full grown galaxies, and galaxy clusters.
Cosmological issues would also be resolved if most of the universe's matter was locked up in WIMPs.

WIMPs may form a dense halo around the Galactic Center.
WIMPs annihilations may produce gamma rays.
VERITAS is sensitive to TeV gamma rays.
VERITAS has spent 108 hours observing the Galactic Center
This work has used an unbinned likelihood analysis to analyize thi data.
This analysis included energy dispersion and psf folding into its likelihood models.

For the Galactic Center analysis, 950 models were assembled.
948 were camera background models.
Another model was added for the point source model for the Galactic Center central source.
The last model was a dark matter halo, using an Einsto cuspy profile, and a $b\bar{b}$ annihilation channel for $m_{\chi}=(4.25,6.5, 10, 14.3, 20.512, 30.4, 45, 66.5, 98) \textrm{TeV}$.
None of these dark matter halos had a significant test statistic, indicating no dark matter halo was detected for any of these $m_{\chi}$.
Upper limits were then calculated by increasing the scale of the dark matter halos until the likelihood had decreased according to a 95\% confidence limit.
This work demonstrates that new limits on the WIMP cross-section can be placed at $m_{\chi}=100\,TeV$ with a cuspy Einasto halo and the $b\bar{b}$ annihilation channel.

This work can be improved in the future by improved background modeling.
Improving the background models would allow for \SIrange{1.5}{4}{TeV} events to be included, providing ~40\% more events to the analysis, providing stronger upper limits for all $m_{\chi}$.

