\cleartooddpage[\thispagestyle{empty}]
\chapter{Veritas}

VERITAS is an Imaging Atmospheric Cherenkov Telescope.
It consists of 4 telescopes, spaced roughly 50m apart.
Each telescope posesses an array of 499 mirrors, and a 499-Photomultiplier-Tube camera out on a boom.
Each mirror focuses gathered photons onto a separate pixel in the camera.
In Section \ref{sec:hardware}, the different hardware components are discussed.
In Section \ref{sec:trig}, the different trigger levels are discussed.
The methods for image reconstruction are discussed in Section \ref{sec:imgrecon}.
In Section \ref{sec:recon}, the reconstruction used in different parts of the analysis are discussed, including the per-telescope image reconstruction in Subsection \ref{subsec:imgrecon}, the direction reconstruction in Subsection \ref{subsec:posrecon}, and the energy reconstruction in Subsection \ref{subsec:enrecon}.


\section{Hardware}\label{sec:hardware}
When a gamma-ray produces an air shower in the atmosphere, the shower emits numerous Cherenkov photons, across nanoseconds in time.
This means that, when the image of the shower reaches the telescopes, only a handful of Cherenkov photons remain, and these photons are spread out over several nanoseconds.
Detection of these limited number of photons over such a tiny span of time are what drive the majority of the design of the VERITAS observatory.

To detect these individual Cherenkov photons, Photomultiplier tubes are used to amplify the signal from single photons.
This produces an analog voltage pulse that then must be digitized.
As the voltage pulses widths are kept to ~ns, to prevent them from overlapping the pulses from photons of other showers, the digization hardware measures each voltage pulse in 1-nanosecond-wide time bins.
Though faint showers may only posess a few photons, brighter showers can still shine several thousand photons onto individual pixels.
This means that the digization hardware must be able to handle a large dynamic range, over several orders of magnitude.
To accomplish this, two amplification levels are used in the digization circuit, High Gain for voltage pulses of a few photons, and Low Gain for voltage pulses of several thousand photons.


4 telescopes
499 pmt's per pixel
~1000V

\section{Triggers}\label{sec:trig}
L1 for pixel
L2 for telescope image
L3 for multi-telescope event : ~300Hz

\section{Event Reconstruction}\label{sec:recon}

\subsection{Image Reconstruction}\label{subsec:imgrecon}

\subsection{Position Reconstruction}\label{subsec:posrecon}

\subsection{Energy Reconstruction}\label{subsec:enrecon}

\section{Sensitivity}
% plot vs energy
Sensitivity is a measure of how difficult it is to get a signifiant detection.
Usually it is measured in 'hours of observation to detect an example point source at 3 sigma'

\section{Effective Area}
% plot of effective area vs energy
Effective area is the measure of how many gamma rays a telescope can detect.
It is referred to as 'Effective' because the detection efficiency is not 100\% at all energies or all parts of the camera.
Instead it is 'effectivly' how large a detection area a telescope has, if it had perfect detection efficiency.
Different telescopes can use this number to compare roughly how many gamma rays can be detected in a given span of time.

\section{Energy Resolution}
% plot of energy resolution
Gamma rays and protons of different energies can produce similar-looking air showers.
Due to this, the reconstruction software cannot perfectly reconstruct gamma rays, which introduces errors into the different qualities of the gamma rays.
This means that when a gamma ray is reconstructed, it has a chance to be reconstructed at a lower or higher energy.
The consequence of this is that gamma rays of a given reconstructed energy can come from a distribution of true simulated energies.
Typically, simulations are run across a number of energies, and an 2D matrix is assembled, where one axis is the true simulated energy, and the other axis is bins of reconstructed energy.


\section{Point Spread Function}
% plot of psf
In addition to the energy being slightly mis-reconstructed, the source position of the gamma ray can also be misreconstructed.
For a gamma ray telescope, a simple measure is to simulate at a single point in the sky, and look at how the reconstructed positions are distributed around that point.
This distribution is referred to as the Point Spread Function.
To first order, most current-generation point spread functions are gaussian distributions.


\section{Energy Sensitivity and Zenith Angle}
% plot of energy sensitivity at 20zen and 65zen

As the telescope points at a lower elevation, the atmospheric volume that can be used to detect air showers greatly increases, leading to an increase in the telescope's effective areafor high energy showers.
The downside is that fewer low-energy gamma rays reach the volume of atmosphere being observed, 

\section{Comparison with Other Observatories}

% sensitivity comparison plots

CTA, HESS, Magic, HAWC, Fermi
